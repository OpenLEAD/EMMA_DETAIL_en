Robots in rails are largely used in industry, where you need to reach various
positions for a task or move a part from place to place on a plant.
In most cases, the robot has a wide space to work and for its installation as
well. For this reason, the base is usually oversized for the robot, ensuring a
big, heavy and rigid plataform and rail. The constraints of the turbine does not
allow parts bigger than $800~mm$ and weight higher than $25~kgf$, for human
transportation.
For its reason the base need to be modular, light and rigid, and flexile in
terms of assembly configurations possibilities.
A modular and flexile concept for robot in a rail is presented by
\cite{Chen2012}, where a curved rail is built around a cylinder structure of a
space station.
Another concept of modular, but flat rail, is presented by \cite{Moon2015}. This
idea is simpler than the curved rail, as you can construct a plane path for the
robot, and have one more degree-of-freedom, as well.

The robot's base is composed by the support, movement and anchorage elements.
The support elements are those that make up the main body of the base and are
responsible for the connecting all elements, providing lift to the whole system.
Movement elements are used to place the arm on the ideal positions for the
process, they  are usually rail and carriage plus some joints on the rail, but
linear actuators, rotational actuators, bearings also fit in this category. For
the anchorage elements, there are two available options: soldering or magnetic
coupling. The magnetic coupling has the advantage of not requiring extra
equipments harming the runner's area. As its effectiveness has been
validated through experimental testing, it's, so, the principal alternative.




% The study of the state of the art of robots for HVOF coating process on
% hydraulic turbine blades covers systems that meet some of the following
% requirements:
% operation in hostile and confined environments; manipulator's end-effector with
% at least 8.5 kg payload, 5 mm precision and $0.67 m/s$ speed; 2.5 m x 3.0 m
% system workspace; and the ability to operate on complex 3D geometry surfaces.
% The robots were organized by fixation technologies.

%O estudo do estado da arte de robôs para a realização de HVOF em pás de
% turbinas hidráulicas contempla os sistemas que atendem a alguns dos requisitos: operar
%em ambientes de alta periculosidade; capacidade de carga para os dispositivos
% HVOF; manipular a pistola HVOF com velocidade de $0.67 m/s$; precisão de 5mm; ter
%área de trabalho de 2.5 m x 2.5 m; e operar sob superfícies 3D de geometria
%complexa. As soluções foram divididas em subseções de acordo com as tecnologias
%de fixação dos robôs.


