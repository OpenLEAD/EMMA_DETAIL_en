\section{Introduction}

%%%%%%%%%%%%

The importance of regular maintenance on hydraulic turbines have already been
established (ref EMMA-SOTA), given that its power output have a near 46\%
increase after maintenance. Providing a meaningful gain for highly dependent
countries like Brasil and Norway.

% Em EMMA-SOTA, é apresentada a importância da manutenção regular das turbinas em
% uma usina hidrelétrica, já que, em sua operação ideal e de máxima eficiência,
% sua potência tem aumento de quase 46\% após manutenção. Aumento significativo,
% principalmente para países dependentes desta forma de energia, como o Brasil e
% Noruega.

The efficiency of a hydraulic turbine is related at some extend to the hydraulic
profile of the runner blades and its degradation is mainly due to two
phenomena: cavitation and abrasion. As a protective measure, the blades are
hardcoated by a process called HVOF (High velocity oxy-fuel coating spraying),
which mitigate the damage caused by the aforementioned effects, but has to be
reapplied periodically.

The hardcoating process can take up to two months per turbine, including turbine
disassemble, blade's hardcoating on a specifically designed environment and
following remounting and calibration.

% A eficiência de uma turbina hidrelétrica depende de inúmeras variáveis, como
% volume de água, queda d'água, o tipo da turbina, o distribuidor e outras. O
% projeto EMMA tem foco na manutenção do perfil hidráulico das pás dos rotores de
% turbinas hidrelétricas, por este se degradar com maior rapidez, exigindo
% manutenções recorrentes. 

% A fim de proteger a pá contra abrasão e cavitação é realizado processo de
% revestimento por asperção térmica, ou, especificamente, a metalização (HVOF).
% Atualmente, este processo pode levar cerca de dois meses por turbina,
% já que exige que a turbina seja desmontada, as pás serem processadas em
% outro ambiente, a turbina seja remontada e recalibrada.

Aiming to reduce the downtime assossiated for the hardcoating process, robotic
\textit{in situ} solutions, i.e. inside the runner environemnt, consisting of a
insdustrial robotic arm mounted over a customized base are scrutinized. This is
the main objetive of the EMMA project, a R\&D project by Fundação Coordenação
de Projetos, Pesquisas e Estudos Tecnológicos (COPPETEC), in partnership with
Rijeza company, Agência Nacional de Energia Elétrica (ANEEL) and Energia
Sustentável do Brasil (ESBR).

Despite the generic approach of this article, the bulb turbine facilities are
different for each power plant. The reference hydroeletric power plant for the
ideas discussed here is the UHE Jirau, near Porto Velho(Brazil - RO). The
Madeira river, where the UHE Jirau is located, has a high concentration of
suspended particles which entails further abrasion on the blades, it also has a
low water level diference of 2 to 20 meter intensifing cavitation. For more
details on the UHE Jirau please reference (EMMA SOTA), however it is important
to recall that there are two main entry points for runner area: the top hatch
with a 35.7 cm diameter, just above the turbine, and the bottom hatch with a 80
cm diameter on the draft tube, 10 m away from the runner and with a 4m access
duct from exterior ground level. But, as a top hatch is not a standard across
other hydroeletric power plants, e.g. the UHE de Santo Antônio near UHE Jirau,
the solutions focus on the bottom hatch.

The workflow for the \textit{in situ} hardcoating can be thought as a
sequence of 4 minor jobs:
Enter the runner area with the robot; Move the robot and anchor it in some
suitable positions near the blade to be hardcoated; Calibrate the robot, in the
sense of identifying the relative positions of the robot, blade and the rest of
the environment; And finally, check if the the robot's arm can cover the whole
blade.

The following sections explore the developments on the EMMA project, they are
organized so to mimic the workflow:
section 2 exposes the ideas for the customized base and related logistics;
section 3 describes the calibration process which has to be done once the robot
is well placed, but before it's able to start coating; section 4 explore the
robot arm's limitations for performing the hardcoating; And lastly section 5
concludes and discusses the future steps for the EMMA project.

% Apesar de o projeto visar uma solução genérica para turbinas bulbo, as
% instalações são diferentes em cada usina. Desta forma, o ambiente de testes
% deste projeto é a Usina Hidrelétrica de Jirau, localizada no Rio Madeira. O Rio
% Madeira carrega muitos sedimentos provocando maior abrasão nas pás, se comparado
% com outros usinas, além disso, a queda d'água de 2 a 20 metros intensifica o
% fenômeno de cavitação. As principais características das instalações da turbina
% em análise estão descritas em EMMA-SOTA, mas vale ressaltar a particularidade
% dos dois acessos principais ao aro câmara, relevantes para a busca de uma
% solução: acesso superior (35.7 cm de diâmetro) e acesso inferior (80 cm de
% diâmetro).

% O projeto EMMA busca uma solução para o processo de metalização \textit{in
% situ}, isto é, revestimento das pás no ambiente da turbina, diminuindo o tempo
% de manutenção e, consequentemente, de máquina parada.  A solução conceitual
% desenvolvida em EMMA-SOTA é a utilização de um manipulador industrial sobre uma
% base. As características do manipulador e da base variam de acordo com o
% acesso: no caso da escotilha superior, a solução é um manipulador industrial de
% pequeno porte e base customizada operada eletronicamente; no caso da escotilha
% inferior, a solução é um manipulador industrial de porte médio e base tipo
% trilho com acopladores magnéticos.
% 
% A análise das instalações da Usina Hidrelétrica de Santo Antônio, em Porto
% Velho, vizinha à Jirau, mostrou que as turbinas não possuem um acesso superior.
% A fim de tentar construir uma solução mais geral, o presente documento visa dar
% continuidade ao projeto, detalhando o estudo de viabilidade técnica para a
% solução da escotilha inferior.

%%%%%%%%%%%%%%%%%%%%%%%%%%%%%%%%%%%%%%%%%%%%%%%%%%%%%%%%%%%%%%%%%%%%%%%%%%%%
% According to the world energy council, hydropower is the most flexible and
% consistent of the renewable energy resources and, at the end of 2008, the total
% capacity of hydropower resources was 874 GW. Brazil is the second
% country in hydropower production, and second with the highest
% consumption of hydropower with a 70.000 MW installed capacity, and 433
% hydroelectric plants in operation. Since Brazil is one of the world's richest
% countries in water resources, and the hydropower is the most dominant across
% the country, it motivates the development and investment in hydropower
% generation., and Brazil is the .
% %\begin{figure}[h!]
% %	\includegraphics[width=\columnwidth]{figs/intro/graph.png}
% %	\caption{Top hydropower producing countries}
% %	\label{fig::cavitacao}
% %\end{figure}
% 
% %O Brasil é um dos países mais ricos do mundo em recursos hídricos, facilitando
% % o desenvolvimento e investimento em geração de energia a partir desse recurso. A
% %energia hidráulica é a mais dominante em todo o país, e o Brasil é o segundo
% %país com maior consumo de energia hidrelétrica no mundo com capacidade
% %instalada de 70.000 MW, 433 usinas hidrelétricas em operação. 
% 
% In Brazil, the renovation and improvement of the built large plants is estimated
% to result in a potential increase of 32.000 MW \citep{goldemberg2007energia}, a
% figure that can be achieved, in large part, by the maintenance of the
% hydropower turbines. These turbines are constantly exposed to abrasion and
% cavitation phenomena, which determine its life cycle.
% %Estima-se que a reforma e melhoria das grandes usinas construídas resultariam
% %em um aumento potencial de 32.000 MW \citep{goldemberg2007energia},
% %número que pode ser alcançado, em grande parte, pela manutenção das turbinas
% %geradoras da energia elétrica. As turbinas estão constantemente expostas aos
% %fenômenos de abrasão e cavitação, os quais determinam sua vida útil.
% 
% The cavitation phenomenon is very well studied and detailed in
% \cite{escaler2006detection}, which outlines their types, occurrences and
% effects in the different hydraulic turbines. This physical phenomenon can cause
% erosions in the hydraulic turbines, leading to water flow instability,
% excessive vibrations and turbine efficiency reduction.
% %O fenômeno de cavitação está muito bem estudado e detalhado em
% %\cite{escaler2006detection}, onde são apresentadas seus tipos, ocorrências e os
% %efeitos nas diferentes turbinas. Esse fenômeno físico pode causar erosões na
% %máquina hidráulica (figura~\ref{fig::cavitacao}), gerando instabilidade de
% % fluxo de água, vibrações excessivas e redução da eficiência da turbina.
% 
% \begin{figure}[h!]	
% 	\includegraphics[width=\columnwidth]{figs/intro/cavitacao2}
% 	\caption{Jirau hydraulic turbine blade eroded by cavitation.}
% 	\label{fig::cavitacao}
% \end{figure}
% 
% Hard coating techniques by thermal aspersion
% are used to reduce the erosion of the turbine's blade from cavitation or
% abrasion, thus increasing its life cycle. This solution is analogous to a paint
% that protects walls from environment exposure. The hard coating procedure is performed
% before the hydraulic turbine installation by a robotic manipulator. The
% procedure requires a robotic system due to high precision, speed, and
% the hazardous substances that are used, as propane and other gases.
% Although sufficient for blade protection, the coating also has a life
% cycle itself, thus it needs to be redone from time to time to ensure the
% blade's protection from physical phenomena.
% %A fim de reduzir o desgaste da pá contra cavitação ou abrasão e aumentar a sua
% %vida útil, utiliza-se a técnica de revestimento por asperção térmica, que pode
% %% ser comparada com uma tinta que protege à exposição com o ambiente. O
% % procedimento é realizado
% %antes da instalação das pás na turbina por um robô, pois exige alta precisão
% %e velocidade, além de expelir substâncias nocivas à saúde. Apesar de suficiente
% % para a proteção da pá, o revestimento também tem vida útil e precisa ser refeito de tempos em tempos para
% %garantir a proteção da pá contra os fenômenos físicos.
% 
% In the specific case of the Jirau hydroelectric dam, built on the Madeira
% river, the number of suspended particles that the river carries intensifies the
% abrasion phenomena, and Rijeza, a hard coating specialized company, identified cavitation erosion on blades, further reducing the coating life cycle.
% Therefore, Jirau hydroelectric dam needs regular maintenance, which,
% in the present situation, would require stoppage of the turbine, removing the
% blades, positioning the blades for coating, coating application, turbine assembling, and recalibration. The downtime to perform all
% maintenance can take up to two months, meaning a huge loss in power generation .
% %No caso específico da usina hidrelétrica de Jirau, construída no rio Madeira,
% %os fenômenos de abrasão são intensos devido ao grande número
% %de partículas que o rio carrega diariamente, reduzindo ainda mais a vida útil
% % do revestimento.
% %Portanto, há a necessidade de manutenção regular, o que, na situação atual,
% %exige paralização da máquina, desmontagem da turbina, posicionamento de cada pá
% %na área designada ao revestimento, aplicação do revestimento, montagem da
% %turbina e recalibração. O tempo de paralização para a realização de
% %toda a manutenção pode levar de um a dois meses, significando uma grande perda
% %na geração de energia. 
% 
% EMMA is an R\&D project by Fundação Coordenação de Projetos, Pesquisas e Estudos
% Tecnológicos (COPPETEC), in partnership with Rijeza company, Agência Nacional de
% Energia Elétrica (ANEEL) and Energia Sustentável do Brasil (ESBR). Its first
% stage is a technical feasibility study of a robotic system to perform
% coating by thermal spray on hydraulic turbine blades within the turbine
% environment. The project aims to significantly reduce the downtime for hard
% coating process.
% %A primeira etapa do projeto EMMA, pesquisa e desenvolvimento
% %realizados pela Fundação COPPETEC, em parceria com a empresa Rijeza, ANEEL e
% %ESBR, é um estudo de viabilidade técnica de um sistema robótico para realizar
% %revestimento por aspersão térmica de turbinas \textit{in situ}, ou seja, dentro
% %do ambiente da turbina (aro câmara). O projeto tem como objetivo reduzir
% %significativamente o tempo de manutenção do revestimento por ser realizado no
% %ambiente confinado da turbina e, portanto, não havendo necessidade de sua
% %desmontagem.
% 
% This document is divided as follows: section 2 describes, in detail, the
% problem, contextualizes the reader in the Jirau environment and
% describes the robot's tasks; section 3 surveys the state of the
% art; section 4 describes the conceptual designs for the robot and mechanical
% bases; finally, the section 5 concludes and outlines the next steps for the
% EMMA project.
% %Este documento está dividido da seguinte forma: a seção 2 descreve
% %detalhadamente o problema, contextualiza o leitor no ambiente da usina de
% %Jirau e descreve as possíveis tarefas do robô; a seção 3 faz um levantamento do
% %estado da arte; a seção 4 descreve os projetos conceituais para o robô; e a
% %seção 5 conclui e descreve os próximos passos para o projeto EMMA. 
