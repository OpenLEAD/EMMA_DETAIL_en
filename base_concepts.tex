\subsection{Base Concepts}

From the conclusion about the base concepts on (EMMA-SOTA), it's clear that rail
based ideas are the simplest solutions. So, focusing on rail concepts, here it's
explored three different ones, differentiated by its degrees of freedom. They
were all designed withstand the dynamic efforts that can cause vibration and
elevated tensions.


$\bullet$~\textbf{Prismatic-Rotacional-Rotacional Base (P-R-R):}
  
  This concept is composed of three degrees of freedom: one prismatic and two
  rotational. The prismatic is simply the rail that shall be aligned with the
  runners rotational axis, being responsible for bringing the arm closer to the
  blade. 
  
  Neste conceito estudou-se a possibilidade de utilizar uma base com $3$ graus
  de liberdade: um prismático e dois rotacionais. O prismático seria composto
  por um trilho alinhado e paralelo ao eixo da turbina que transportaria o robô
  até a região próxima a pá. Uma junta rotacional e com eixo vertical orientaria
  a base nesta direção e uma junta perpedincular à primeira faria o
  posicionamento da base do robô para então iniciar o processo de revestimento.
  A figura~\ref{fig::base_prr} ilustra este conceito.
    
  \begin{figure}[h!]
   \centering
   \includegraphics[width=0.8\columnwidth]{figs/bases/base_prr}
   \caption{Base Primático-Rotacional-Rotacional}
   \label{fig::base_prr}
\end{figure}

  A vantagem deste conceito é conferir um alcance grande ao manipulador através
  da base, permitindo que este possa ser de menor alcance próprio, mas ao mesmo
  tempo mais leve.
  Porém, devido à configuração de juntas e pelos resultados encontrados no
  estudo cinemático, a manobrabilidade desta base seria reduzida naquele espaço,
  havendo posicionamentos difíceis de serem alcançados, ou até impossíveis
  dependendo do manipulador escolhido.
  
$\bullet$~\textbf{Base Prismática (P):}

  Este conceito consiste de um trilho (junta prismática) para o transporte do
  manipulador desde a escotilha até o ponto de interesse para revestimento na
  face anterior ou posterior da pá. Quando posicionado, remove-se a seção
  do trilho na direção que obstrui a rotação do rotor. Neste conceito,
  adiciona-se um grau de liberdade ao sistema utilizando a própria rotação do
  rotor, posicionando a pá em relação ao robô. A base mecânica então forneceria
  apenas movimento no trilho na direção do eixo da turbina, deixando fixas as
  outras direções. O procedimento para o revestimento seria o posicionamento do
  rotor, deixando a região a ser processada ao alcance do manipulador; o
  posicionamento do robô no trilho, em relação a pá; a ancoragem do robô
  no ambiente; e o revestimento da região possível para aquela posição.
  Repete-se então este procedimento até ter toda a face processada e
  posiciona-se a próxima pá para revestimento, sem necessidade de mover ou
  desmontar a base do robô até todas as faces daquele lado estarem completas.  A
  figura~\ref{fig::base_p} ilustra este conceito.
  
  \begin{figure}[h!]
   \centering
   \includegraphics[width=0.8\columnwidth]{figs/bases/base_p}
   \caption{Base Prismática}
   \label{fig::base_p}
\end{figure}
  
  Este conceito foi estudado para o manipulador MH$12$, que de acordo com a
  análise cinemática consegue processar toda a extensão vertical. Para outros
  manipuladores, seria necessário incluir uma junta prismática, adicionando um
  grau de liberdade, na direção vertical.
  
  A análise cinemática também demonstrou que seriam necessárias muitas posições
  do rotor para completar uma face da pá. Há inclusive dificuldades operacionais e
  de segurança no procedimento de rotação do rotor que devem ser considerados. O
  rotor só pode ser girado manualmente, não fornecendo precisão no
  posicionamento da pá em relação a base. Por ser uma tarefa manual, deve-se ter
  procedimentos adequados de segurança para preservar tanto o operador quanto os
  equipamentos próximos. Estas preocupações tornam a solução pouco prática sob o
  ponto de vista operacional.

$\bullet$~\textbf{Base Prismática-Rotacional-Prismática (P-R-P):}

  Este conceito consiste de uma base composta por um trilho primário (junta
  prismática $1$), uma plataforma de base pivotada por mancal e rolamentos entre
  o trilho primário e secundário (junta rotacional) e um trilho secundário
  (junta prismática $2$). Montado o trilho primário alinhado ao eixo da turbina
  a base rotacional sobre o trilho primário, fixa-se o robo sobre a base
  rotacional. Esta base permitrá a montagem do trilho secundário apenas quando o
  robô atingir a região de interesse para revestimento. Quando posicionado o
  manipulador, monta-se então o trilho secundário alinhado ao plano paralelo a
  face da pá e ancora-se a base no ambiente. Desta forma, o robô pode-se
  movimentar ao longo de toda a extensão da pá por meio do trilho secundário e
  também se aproximar e se afastar da superfície da pá, por meio do trilho
  primário. A figura~\ref{fig::base_prp} ilustra este conceito.

\begin{figure}[h!]
   \centering
   \includegraphics[width=0.9\columnwidth]{figs/bases/base_prp}
   \caption{Base Primática-Rotacional-Prismática}
   \label{fig::base_prp}
\end{figure}

  Desta forma, o rotor deve estar girado em, no mínimo $30^o$ para não haver
  contato com o trilho primário. A análise cinemática será realizada para
  encontrar a melhor configuração de juntas da base que permite ao robô se
  movimentar nos graus de liberdade da base, sem alterar o posicionamento do
  rotor e, assim, cobrir uma face inteira da pá. Para a repetição do processo
  nas outras pás do lado da sucção da turbina, é necessária a desmontagem do
  trilho secundário, o recuo do robô e desmontagem de parte do trilho primário,
  permitindo o giro do rotor para a pá seguinte.
  Para as faces do lado de adução, não é necessária a desmontagem parcial do
  trilho primário. 
