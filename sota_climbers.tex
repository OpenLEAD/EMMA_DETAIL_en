\subsection{Climbing robots}\label{sota_climbers}
Climbing robots are systems capable of supporting its own weight against
gravity, moving in simple or complex geometric structures, such as
walls, ceilings and roofs, turbine blades and nuclear plants.
Climbers provide operational efficiency in hazardous environments, and increase
operators health and safety. Some applications for climbing robots are:
skyscrapers inspection and cleaning, storage tanks diagnosis in nuclear
power plants, shell ships and turbines welding and
maintenance \citep{armada2003application}.
%Robôs escaladores são sistemas capazes de sustentar seu próprio peso contra a
%gravidade, movendo-se em simples ou complexas estruturas geométricas, como
%paredes, tetos e telhados, palhetas de turbinas e plantas nucleares.
%Essa classe de robôs oferece eficiência operacional em ambientes
%de alta periculosidade, sendo utilizados visando saúde e segurança dos
%trabalhadores, como em inspeção e limpeza de arranha-céus, diagnóstico de
%tanques de armazenamento em plantas nucleares, solda e manutenção de cascos de
%navios e palhetas de turbinas \citep{armada2003application}. 

The development major challenges for climbers are mobility, adhesion, power
consumption, load capacity, and weight. In \cite{modular} and \cite{climbsurv},
climbers are divided into types of locomotive mechanisms: legs; walker; translation; wheels; tracks;
advance by arms; cable-driven; and biomimetics. And adhesion types:
suction or pneumatic; magnetic; electrostatic; chemical; gripping; and hybrid.
%Os grandes desafios nos projetos de sistemas escaladores são mobilidade e
%aderência, além de consumo de energia, capacidade de carga e peso. Em
%\cite{modular} e \cite{climbsurv}, os robôs escaladores são divididos em tipos
%de locomoção:
%pernas; como andador; utilizando segmentos deslizantes; rodas; esteiras; avanço
%pendurado por braços; por cabos; e biomimética. E categorias de adesão: sucção
%ou pneumática; magnética; eletrostática; química; preensão; e híbrida.

In the specific case of the HVOF coating problem in hydropower turbines, the
following climber robots should be investigated, as all of them have millimetric
accuracy, and robustness:
%No caso específico deste estudo da arte, destacam-se os robôs escaladores com
% as seguintes aplicações:

\begin{itemize}
  \item \emph {ships and turbines}: RRX3 for welding
   \citep{rrx3}, \emph{Climbing Robot for Grit Blasting for cleaning}
   \citep{crgb} and ICM Robot for inspection \citep{icm};
  \item \emph{Industrial}: ROME II \citep{roma} and CROMSCI \citep{CROMSCI}, both for inspection;
  \item \emph{petrochemical plant}: TRIPILLAR \citep{tripillar} for inspection.
\end{itemize}

The RRX3, Daewoo Shipbuilding \& Marine Engineering, is a
robot for hull ship welding with manipulator. Its adhesion is gripping
type, locomotion is translation type (sliding segments), and longitudinal locomotion by wheels.
The RRX3 robot has a 1.5 m manipulator with three prismatic joints and three
revolution joints  (3P3R) for welding operation. The system weighs 120 kg with 5
kg payload, it has a manipulator with welding tool, but low speed end effector.
Regarding the HVOF application, the locomotion type is not efficient.
%O RRX3 (figura~\ref{rrx3}), Daewoo Shipbuilding and Marine Engineering, é um
%robô para a soldagem de casco de navios. Possui adesão por preensão, locomoção
% transversal utilizando segmentos deslizantes e locomoção longitudinal por rodas. Possui um manipulador de 1.5 m com três juntas
%prismáticas e três juntas de revolução (3P3R) para a operação de soldagem. 

%RRX 3 main characteristics are: base and manipulator with
%120 kg and 5 kg of payload, respectively; accurately
%millimeter manipulator and low speed end effector; robustness; welding tool
%operation; limited translation type locomotion.
%As características principais do robô são: base e manipulador com
%capacidades de carga de 120 kg e 5 kg, respectivamente; manipulador com
% precisão milimétrica e efetuador de baixa velocidade; robustez para operar em ambiente de
%alta periculosidade; opera instrumento de solda; e locomoção transversal é
%restrita à aplicação.

%\begin{figure}[ht]
%\centering
%\includegraphics[width=8.4cm]{figs/climbers/RRX3_moving.jpg}
%\caption{RRX3 translation locomotion.}
%\label{rrx3}
%\end{figure}

The \emph{Climbing Robot for Grit Blasting}, %(figura~\ref{grit}), 
University of Coruna, is a robot for ship abrasive blasting. The robot moves by
two sliding platforms with magnetic adhesion. The platforms have relative motion
between them and can rotate to compensate ship's hull curvatures or to
deflect objects. The abrasive system is similar to HVOF, but the robot locomotion not applicable to complex structures.
%O \emph{Climbing robot for Grit Blasting} (figura~\ref{grit}), University of
%Coruna, é um robô para jateamento abrasivo em navios. O robô utiliza duas
% plataformas deslizantes com sistema de adesão por ímã magnético. Os módulos apresentam movimentação relativa entre si e pode rotar
%para compensar as curvaturas do casco do navio ou desviar de objetos. 

%The \emph{Climbing robot for Grit Blasting} main characteristics are:
%abrasive system similar with HVOF; accurately
%millimeter movement; locomotion not applicable to complex structures; no
%manipulator.
%As características principais do robô são: base com
%capacidade de carga de sistema abrasivo semelhante a HVOF; base com
%locomoção de precisão milimétrica; locomoção ampla, mas não aplicável a
%estruturas complexas; e não possui manipulador, sendo necessário percorrer todo
%o casco.

%\begin{figure}[ht]
%\centering
%\includegraphics[width=8.4cm]{figs/climbers/grit.png}
%\caption{Climbing robot for Grit Blasting}
%\label{grit}
%\end{figure}


\emph{The Climber}%(figure~\ref{icm})
, ICM Robotics, is an inspection robot for
wind turbines, coating removal, surface cleaning and coating application.
It has pneumatic adhesion and locomotion by tracks. It has 25 kg base payload,
and a small sized low speed manipulator. The locomotion type presents
restrictions to some curvatures.
%\emph{The Climber} (figura~\ref{icm}), ICM Robotics, é um robô para inspeção de
%turbinas eólicas, remoção de revestimento, limpeza de superfície, e aplicação
% de revestimento.
%Possui adesão pneumática (sucção) e locomoção por esteiras. 

%\emph{The Climber} main characteristics are: 25 kg base payload; accurately
%millimeter movement; a modular manipulator can be attached to the base; small
%size manipulator and low speed; locomotion type presents restriction to
%some curvatures.
%As características principais do robô são: base com capacidade de carga de 25
%kg; base com locomoção de precisão milimétrica; manipulador modular pode ser
%acoplado à base; manipulador de dimensão reduzida e baixa velocidade; e
%locomoção apresenta restrição a algumas curvaturas acentuadas.

%\begin{figure}[ht]
%\centering
%\includegraphics[width=8.4cm]{figs/climbers/icm.png}
%\caption{The Climber}
%\label{icm}
%\end{figure}

The Rome II%(figura~\ref{roma2})
, University Charles II of Madrid, is an inspection robot for complex
environments. It has pneumatic adhesion and moves like a caterpillar
(biomimicry). Rotation and planning trajectory are performed optimally to
ensure stability and obstacle avoidance.
%O ROMA II (figura~\ref{roma2}), Universidade Carlos II de Madrid, é um robô
% para inspeção de ambientes complexos. A sua tecnologia de adesão é pneumática (sucção) e
%locomove-se como uma lagarta (biomimética). Sua movimentação e planejamento de
%trajetória são realizados de maneira ótima de forma a garantir estabilidade e
%evitar obstáculos. 

%The Rome II main characteristics are: high payload capacity; accurately
%millimeter movement; no manipulator; locomotion for complex environments.
%As características principais do robô são: base com grande capacidade de carga;
%base com locomoção de precisão milimétrica; não possui manipulador; locomoção
% em ambientes de grande complexidade.

%\begin{figure}[ht]
%\centering
%\includegraphics[width=8.4cm]{figs/climbers/roma2.jpg}
%\includegraphics[width=4.2cm,height=4.2cm]{figs/climbers/roma2.jpg}
%\caption{ROMA II.}
%\label{roma2}
%\end{figure}
CROMSCI% (figure~\ref{cromsci})
, Kaiserslautern University of Technology, is an
inspection and autonomous robot for large concrete walls, as
pillars of bridges and dams. Its adhesion system is composed of seven vacuum
chambers (suction), valves and pressure sensors for system control. The
locomotion system has omnidirectional wheels.
%CROMSCI (figura~\ref{cromsci}), Kaiserslautern University of Technology, é um
%robô autônomo para inspeção de grandes paredes de concreto, como pilares de
% pontes, barragens. Seu sistema de adesão é composto por sete câmaras de vácuo (sucção), com um sistema
%de controle por válvulas e sensores de pressão para reagir rapidamaente a
%condições adversas. Locomove-se com rodas omnidirecionais para locomoção.

%The CROMSCI main characteristics are: low payload capacity; accurately
%millimeter movement; no manipulator; low speed.
%As características principais do robô são:
%base com pouca capacidade de carga; base com locomoção de precisão milimétrica;
% não possui manipulador; e apresenta baixa velocidade.

%\begin{figure}[ht]
%\centering
%\includegraphics[width=8.4cm]{figs/climbers/cromsci.jpg}
%\caption{The CROMSCI robot.}
%\label{cromsci}
%\end{figure}

TRIPILLAR, Ecole Polytechnique Federale de Lausanne, is a small inpection robot
(96 x 46 x 64 mm) for petrochemical plants. Its adhesion is done by magnetic
legs on a caterpillar triangular shape, and it moves by tracks.

A specific case of climbers is the \textit{cabling robots}, which use a set
of cables to ensure its proper positioning in its working area. The cables provide
manipulator range improvement, decrease the adhesion complexity and reduce the
weight carried by the robot. As an example, the \textit{torboMate} is a climber
with magnetic adhesion, it can have two or more emitting jets with 4000 bar of
supply capacity. It has 45 kg and reaches 20 m/min speed \citep{torbo}.

RIWEA is a purely cabling robot, as it has no other type of position
adjustment, for wind power turbines cleaning. It is an open frame concept robot
which uses four ropes to move up and down. It has five main parts, which
automatically adjust to the blade surface during its move
\citep{jeon2012maintenance}. Its greatest strength lies the ability to adapt the
curvature of the blade while maintaining a foothold on it, and it is also less
susceptible to vibration \citep{riwea}.

Climbing robots are widely applicable, have different adhesion solutions and
mobility. There is not, so far, a climber that fulfills all the HVOF
requirements for the turbine blade coating, but some of the systems, such as
\emph{The Climber} (ICM Robotic), can generate complete solutions with
adaptations.
%Os robôs escaladores são utilizados em diversas aplicações e possuem diferentes
%soluções de aderência e locomoção, como foi exposto nesta subseção. Não há,
%até o momento, um robô escalador que possui todas as características
%exigidas para a tarefa de HVOF em pás de turbinas, porém a adaptação de
%alguns desses sistemas, como \emph{The Climber} da ICM Robotic, pode gerar
%soluções completas.

The advantages for climbing robot solution are: easily installation, small sized
manipulator, small base, lightweight, autonomy; and the disadvantages are:
complex locomotion system, complex mechanics, manual installation on blade,
well-developed robot safety system, limited battery or umbilical management
system. 
%As vantagens e desvantagens para solução de robôs escaladores são:
%\textbf{Advantages:}
%\begin{itemize}
%  \item Easily installation;
%  \item Small size manipulator, since robot moves on blade;
%  \item Small base;
%  \item Small weight;
%  \item Autonomy while operating; 
%\end{itemize}
%and disadvantages 
%\textbf{Disadvantages:}
%\begin{itemize}
%  \item Complex locomotion system with obstacle avoidance and path
%  planning;
%  \item Complex mechanics, as robot should be able to support its weight plus
%  the manipulator and the HVOF spary gun;
%  \item Robot must be manually installed on each blade or a complex locomotion
%  system by arms should be developed;;
%  \item Robot safety system must be well developed;
%  \item Limited battery or umbilical management system for mobile robots;
%\end{itemize}