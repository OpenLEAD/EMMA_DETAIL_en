\subsubsection{Design of a mobile robot and rail fixed on blade}\label{proj_rail}
The robotic manipulator and rail base fixed on blade satisfie all
requirements for the HVOF coating and the inspection process. The development of
a compact system for easy transportation and its installation on the arc
chamber are possible, since the manipulator dimensions are reduced due to the
extra mobility provided by the rail.

%A utlização de um manipulador robótico sobre trilhos satisfaz todos os
%requisitos para a realização de um processo de inspeção e metalização
%utilizando a técnica HVOF. O desenvolvimento de um sistema compacto para o
%transporte através do acesso pela escotilha inferior e sua instalação no aro
%câmara da turbina são possíveis, pois as dimensões do manipulador podem ser
%reduzidas por meio da mobilidade extra proporcionada pela introdução do trilho.

In the context of the proposed application, the solution consists in a system
similar to Roboturb, presented in section \ref{sec::rail}. Thus the rail should
be flexible to be able to follow the blade curvature, it should allow several placement options,
and, as the blade does not have a high magnetic permeability, the adhesion would
be by active suction cups with specific material to support large temperature
variations.

%No contexto da aplicação proposta, foram concebidas duas possibilidades para a
%fixação do sistema de trilhos. A primeira solução consiste em um sistema
%semelhante ao Roboturb, apresentado na seção \ref{sec::rail}. O sistema
% proposto se trata de um manipulador robótico com fixação diretamente na pá da
%turbina. O trilho deverá ser flexível para ser capaz de acompanhar a curvatura
%da pá e possibilitar diversas opções de posicionamento. Como o material da pá
%não possui alta permeabilidade magnética (Inox 420), a solução de fixação seria
%por ventosas ativas e com material específico para suportar as grandes
%variações de temperatura que a pá pode alcançar (temperatura ambiente a
%$100^oC$ durante a metalização).

%Uma abrangente pesquisa de robôs comerciais industriais de pequeno porte
% apontou que há manipuladores com carga entre 12 e 20 kg e velocidade necessários,
%sendo o LBR da Kuka o que possui melhor benefício peso/alcance, 30 Kg e 820 mm,
%respectivamente. %Para este manipulador, a metalização deverá ser
%realizada em, pelo menos, quatro etapas com quatro trilhos diferentes e
%customizados, e placas de sacrifício para evitar mau aplicação da metalização
%durante as trocas de sentido na movimentação do robô.

%A fixação de um trilho na pá apresenta diversas complexidades, como: a
%necessidade de manualmente instalar/desinstalar o sistema trilho/robô diversas
%vezes em cada pá; o projeto do trilho customizado e flexível; e ventosas ativas
%especiais que suportam variação de temperatura.




\textbf{Solution conclusion}

Fixing a rail on the blade has some complexities: rail and robot manual
installation/uninstallation for each blade side; design of customized flexible
rail; and design of special active suction cups that support temperature
variation. It is possible to use an industrial manipulator, such as the
Kuka lightweight (30 kg), making the design focused on signal processing,
mapping, localization, and control, and the rail construction. 

%A solução com trilho externo se mostrou vantajosa em comparação ao robô em
%trilho customizado acoplado à pá, devido à complexidade e intervenções
%manuais. Há a possibilidade de utilizar um manipulador industrial, tornando o
%foco do projeto em processamento de sinais, mapeamento, localização e controle,
%além da construção do trilho. Porém, a montagem da estrutura e a instalação de
%todo o sistema atrás da pá podem ser custosas, sendo esta ainda uma solução
%considerada complexa.