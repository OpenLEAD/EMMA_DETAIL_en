
\subsection{Sensor's Survey}

The use of a spatial data acquisition sensor is not limited calibration, it may
also be useful to retrieve the model of the blade's hydraulic profile,
the ideal profile as well as the current state of the blade.

A survey on the sensors that best fit this purpose is organized by type
below.

% A utilização de um sensor de aquisição de dados espaciais não se limita somente
% a localização, mas, dependendo do sistema a ser escolhido, pode também ser útil
% na reconstrução do modelo do perfil hidráulico da pá, tanto do perfil ideal
% quanto do estado atual da pá a ser processada (Tarefas descritas em
% \ref{sec::introducao}).

% Esta seção irá apresentar os segmentos de sensores que possam suprir essa
% necessidade, assim como suas vantagens e limitações. 


\subsubsection{3D Scanners}

3D scanners are high-precision equipment used in industry usually in metrology
applications, construction, strains monitoring, among others. The equipment
consists of a laser beam which the direction is given by the angle of a
controlled mirror.
The distance to an object is calculated from the phase change between the
emitted and reflected signal. Its price is in the range of U\$ 70,000.00

% 3D scanners são equipamentos de alta precisão utilizados na indústria
% geralmente em aplicações de metrologia, construção civil, monitoramento de
% deformações, entre outras. O equipamento consiste em um feixe de laser que é
% direcionado por meio de um espelho e a partir da mudança de fase do sinal
% refletido é possível calcular a distância até o objeto atingido. O seu preço
% esta na faixa de U\$70.000,00.

\paragraph{Total Station}


This type of sensor works from a single laser beam which is directed by a
rotating mirror coupled to a high resolution stepper motor. The sensor also has
a swivel base, thus it is able to perform a $360^o$ scan on the environment.
Thanks to its construction, this type of sensor has a high density of points
and resolution, but its refresh rate is not much high .

% Esse tipo de sensor funciona a partir de um único feixe laser que é direcionado
% por um espelho giratório acoplado a um motor de passo de alta resolução. O
% sensor possui também uma base giratória, realizando assim um escaneamento de
% $360^o$ do ambiente. Graças a sua construção, esse tipo de sensor possui uma
% alta densidade de pontos e resolução, porém sua taxa de atualização não é muito
% alta.

%\renewcommand{\arraystretch}{1.2}
% \begin{table}[b]
% \begin{minipage}{\textwidth}
% \centering
\begin{center}
\begin{tabular*}{\columnwidth}{l @{\extracolsep{\fill}} ccc}
\hline
{\bf Feature}           &
{\bf\begin{tabular}[x]{@{}c@{}}Faro Focus\\X 330\end{tabular}} & {\bf Leica
P16} & {\bf\begin{tabular}[x]{@{}c@{}}Nikon \\MV330\end{tabular}}   \\[1mm]
\hline 
{\bf Vertical FoV\footnotemark[1] }%
% \begin{tabular}[c]{@{}l@{}}Vertical\\
% FoV\end{tabular} }
& $300^o$ & $270^o$ & $45^o$
\\[3.5mm]
{\bf Horizontal FoV\footnotemark[1]}%\begin{tabular}[c]{@{}l@{}}Horizontal\\
% FoV\end{tabular} }
& $360^o$ &$360^o$ & $360^o$
\\[3.5mm] {\bf Reach}            & 330m & 40m & 30m  \\[1mm]
{\bf \begin{tabular}[c]{@{}l@{}}Scan Speed\\ (\textit{k}pts/s)\end{tabular}}
                   & 976  & 1,000 & 2 \\[3.5mm]
{\bf Precision}         & $\pm$2mm & $\pm$3mm  & $\pm$0.5mm \\[1mm]     
{\bf Weight}  & 5.2 kg  & 12.25 kg & \textit{N.A.} \\[1mm]
{\bf Size (cm)}  & 24x20x10  & 24x36x40 & \textit{N.A.} \\[1mm] 
{\bf Batery Life} & 4.5  & 5.5  & \textit{N.A.} \\[1mm]
{\bf \begin{tabular}[c]{@{}l@{}}Working\\ Temperature\end{tabular}} &
$5^o$C/$40^o$C & $-20^o$C/$50^o$C & \textit{N.A.} \\[3.5mm] {\bf
\begin{tabular}[c]{@{}l@{}}Price\\ U\$1,000.00\end{tabular}} & 70 & 80 & 355  
\\[1mm]
\hline
\end{tabular*}
\captionof{table}{Comparativo Luz Estruturada vs ToF. Fonte:}
%\caption{Dados principais do processo de metalização HVOF}
%\label{tab::estructvstof}
\end{center}
% \end{minipage}
% \end{table}
 \footnotetext[1]{FoV: Field of View}
% \enlargethispage{-4\baselineskip}
% 
% \begin{itemize}
%   \item \textbf{Faro Focus X 330}
%   	\begin{itemize}
%   		\item Campo de visão (vertical/horizontal): $300^o$/$360^o$
%   		\item Alcance: 330m
%   		\item Velocidade máx. de escaneamento: 976.000pts/s
%   		\item Precisão: $\pm$2mm
%   		\item Peso: 5,2 kg
%   		\item Tamanho: 240 x 200 x 100 mm
%   		\item Vida da bateria: 4,5 horas
%   		\item Temperatura ambiente: $5^o$ - $40^o$ C
%   		\item Preço: U\$70.000,00
% 	\end{itemize}
%   \item \textbf{Leica P16}
%   	\begin{itemize}
%   		\item Campo de visão (vertical/horizontal): $270^o$/$360^o$
%   		\item Alcance: 40m
%   		\item Velocidade máx. de escaneamento: 1.000.000pts/s
%   		\item Precisão: $\pm$3mm
%   		\item Peso: 12,25 kg
%   		\item Tamanho: 238 x 358 x 395 mm
%   		\item Vida da bateria: 5,5 horas
%   		\item Temperatura ambiente: $-20^o$ - $50^o$ C
%   		\item Preço: U\$80.000,00
% 	\end{itemize}
%    \item \textbf{Nikon MV330}
%   	\begin{itemize}
%   		\item Campo de visão (vertical/horizontal): $45^o$/$360^o$
%   		\item Alcance: 30m
%   		\item Velocidade máx. de escaneamento: 2000pts/s
%   		\item Precisão: $\pm$0.5mm
%   		\item Peso: \textit{não informado}
%   		\item Tamanho: \textit{não informado}
%   		\item Vida da bateria: \textit{não informado}
%   		\item Temperatura ambiente: \textit{não informado}
%   		\item Preço: U\$355.000,00
% 	\end{itemize}
% 
% 
% \end{itemize}


\begin{figure}[h!]
	\includegraphics[width=0.6\columnwidth]{figs/3dsensors/faro}
	\caption{Faro Focus X330}
    \label{fig::faro_focus}
\end{figure}

%TODO acertar figura

\paragraph{Velodyne}

% TODO ELAEL reler
The company Velodyne currently has three models of 3D LIDAR . Models range
basically in the number of pairs of transmitters and receivers and hence the
final resolution. The models are VLP-16, HDL-32E  and HDL-64E, with 16,32 and 64
channels respectively . This type of sensor has a high rate of update, however
does not have a high density data, as expected by the usual trade off \textit{Density
vs Update Rate}.
The most widespread used model is the intermediary HDL- 32E. It has internal
MEMS\footnotemark[2](accelerometers and gyroscopes) for bearing correction and
a proctection rate IP67
A summary of its features:

% A empresa Velodyne possui, atualmente, 3 modelos de 3D Lidar. Os modelos variam
% basicamente no número de pares de emissores e receptores e, consequentemente, na
% resolução final. Os modelos são o VLP-16, o HDL-32E e HDL-64E, com 16,32 e 64
% canais respectivamente. Esse tipo de sensor possui uma alta taxa de
% atualização, entretanto não possui uma alta densidade de dados. O modelo mais
% utilizado é o intermediário HDL-32E.

\footnotetext[2]{MEMS: Microelectromechanical systems}

% \begin{itemize}
% \item 32 pares laser/detector  
% \item Campo de Visão: +10.67$^o$ to -30.67$^o$ (vertical)
% \item Rotação de $360^o$
% \item Alcance - 1m - 100m 
% \item 10 Hz frame rate (selecionável 5-20Hz)
% \item Temperatura de Operação $-10^o$ to $+60^o$ C
% \item Acurácia: $<$2 cm
% \item Resolução Angular (vertical) 1.33$^o$
% \item Peso: HDL-32E = 1kg; Cabos = 0.3kg
% \item Tamanho: 15cm altura x 8.6cm diâmetro
% \item Proteção: IP67
% \item Correção de orientação (internal MEMS acelerometros and gyros)
% \end{itemize}

\begin{center}
\begin{tabular*}{\columnwidth}{l @{\extracolsep{\fill}} c}
\hline
{\bf Feature}           &
{\bf Velodyne HDL-32E}   \\[1mm]
\hline 
{\bf Vertical FoV\footnotemark[1] }%
& $+10.67^o$ to $-30.67^o$
\\[3.5mm]
{\bf Horizontal FoV\footnotemark[1]}
& $360^o$ 
\\[3.5mm]

{\bf Reach}            & 1m - 100m  \\[1mm]
{\bf \begin{tabular}[c]{@{}l@{}}Scan Speed\\ (frames/s)\end{tabular}}
                   & 10 (user-selectable 5-20) \\[3.5mm]
{\bf Precision}         & $<$2 cm \\[1mm]     
{\bf Weight}  & 1kg (+ 0.3 kg cables) \\[1mm]
{\bf Size}  & 15cm height x 8.6cm diameter \\[1mm] 
{\bf \begin{tabular}[c]{@{}l@{}}Working\\ Temperature\end{tabular}} &
$-10^o$C/$+60^o$C \\[3.5mm]
\hline
\end{tabular*}
\captionof{table}{Comparativo Luz Estruturada vs ToF. Fonte:}
%\caption{Dados principais do processo de metalização HVOF}
%\label{tab::estructvstof}
\end{center}


\begin{figure}[h!]
   \centering
   \includegraphics[width=0.8\columnwidth]{figs/3dsensors/velodyne}
   \caption{Velodyne Models}
   \label{fig::velodyne_models}
\end{figure}

\paragraph{Forecast 3D Laser System}

The Forecast 3D is composed by a SICK's 2D laser scanner, models LMS 151 or 511,
attached to a \textit{pan\&tilt} unit. It is a less expensive archtecture with a
price around U\$37.000,00.

% O sensor Forecast 3D consiste em um senor 2D laser da SICK, modelo LMS 151 ou
% 511, acoplado a uma unidade $pan-tilt$. O seu preço esta na faixa de
% U\$37.000,00.


\begin{figure}[h!]
   \centering
   \includegraphics[width=0.8\columnwidth]{figs/3dsensors/forecast}
   \caption{Forecast 3D Laser System}
   \label{fig::forecast}
\end{figure}

\subsubsection{ToF Cameras}

Time-of-Flight Cameras are comprised of a special camera only. This type of
device uses an internal infrared source and, similarly to laser devices, the
distances are calculated from the difference of phase between the original and
the reflected signal.
Despite enabling the calculation of each object's distance simultaneously  in
the illuminated region by the IR source, this technology has limited resolution.

% Conhecidas como Time-of-Flight Cameras, são dispositivos compostos por apenas
% uma câmera, não necessitando de uma configuração estéreo para triangularização
% de imagens. Esse tipo de dispositivo utiliza uma fonte infra-vermelho interna e de
% forma análoga aos dispositivos laser, calcula a distância a partir da diferença
% de fase do sinal refletido. Entretanto, essa tecnologia possibilita o cálculo
% simultâneo das distâncias de cada objeto na região iluminada pela fonte IR,
% mesmo que com resoluções limitadas.

% \paragraph{Mesa Imaging SwissRanger SR4000}
% 
% 
% \begin{itemize}
%   \item Maximum Framerate: 50 FPS
%   \item Alcance para detecção: 0.1 - 10.0 m
%   \item Alcance calibrado: 0.8 - 8.0 m
%   \item Temperature Drift (T) - $\leq$ 1.5 mm/$^o$C (max.) - For 10$^o$C
%   $\leq$ T $\leq$ 50$^o$C
%   \item Size: 65 x 65 x 76 mm
%   \item Weight: 510 g
% \end{itemize}
% 
% \begin{figure}[h!]
%    \centering
%    \includegraphics[width=0.8\columnwidth]{figs/3dsensors/mesa2}
%    \caption{Mesa Imaging SwissRanger SR4000}
%    \label{fig::mesa}
% \end{figure}

\paragraph{ Argos 3D - P100}
%http://www.bluetechnix.com/en/products/depthsensing/product/sentis-tof-m100/%
\begin{itemize}
  %\item Medidas de distância e vídeo em tons de cinza
  \item Resolution: 160 x 120 pixels
  \item Up to 160 fps
  \item Application Range: from 0.1 to 3m%$>$3m  (extensível até 10m indoor)
  \item Field-of-View: $90^o$
  \item Dimensions: 26 x 75 x 57 mm
  \item Weight: 140g
\end{itemize}

\begin{figure}[h!]
   \centering
   \includegraphics[width=0.8\columnwidth]{figs/3dsensors/argos3dp100}
   \caption{Sensor Argos 3D - P100}
   \label{fig::forecast}
\end{figure}

\subsubsection{Structured-Light 3D Scanner}

These sensors are composed of a source of infrared and a receiver.
A pattern is projected onto the scene to be reconstructed and, from its
distortion, the calculation of distances is possible.
% 
% Estes sensores constituem de uma fonte emissora de infra-vermelho e um receptor.
% Um padrão é projetado na cena a ser reconstruida e a partir da distorção desse
% padrão é possível o cálculo de distâncias. 

%TODO exemplos dos sensores de luz estruturada
%TODO Pros e cons

%TODO ELAEL - decidir se abre uma subseção d eaplicações ou coloca um exemplo de
% aplicação em cada componente - utilizar o seu material do SOTA em 3D sensors. 

\begin{center}
\begin{tabular*}{\columnwidth}{l @{\extracolsep{\fill}} cc}
\hline
{\bf \bf\begin{tabular}[x]{@{}c@{}}Technical\\Features\end{tabular}}           & {\bf\begin{tabular}[x]{@{}c@{}}Structured-Light\\3D Scanner\end{tabular}} & {\bf ToF}                                               \\ \hline
{\bf Software Complexity}  & Average                                        &
\cellcolor[HTML]{92D050}{\color[HTML]{000000} {\bf Low}} \\ {\bf Cost}          
& \cellcolor[HTML]{FE0000}{\color[HTML]{FFFFFF} {\bf High}}  & Average     \\
{\bf Size}                   & Big                                             
& \cellcolor[HTML]{92D050}{\bf Small}                      \\
{\bf Response time}         & \cellcolor[HTML]{FE0000}{\color[HTML]{FFFFFF} {\bf
High}} & \cellcolor[HTML]{92D050}{\bf Low}                        \\
{\bf Depth accuracy}  & \cellcolor[HTML]{92D050}{\bf High}                         & Average                                                      \\
{\bf Low light quality}  & \cellcolor[HTML]{92D050}{\bf High}                         
& \cellcolor[HTML]{92D050}{\bf High}                          \\{\bf Bright
light quality} & \cellcolor[HTML]{FE0000}{\color[HTML]{FFFFFF} {\bf Low}} &
\cellcolor[HTML]{92D050}{\bf High}                          \\
{\bf Energy consumption}        & Average                                                     
& \cellcolor[HTML]{92CDDC}{\bf Scalable}                    \\
{\bf Reach}                   & \cellcolor[HTML]{92CDDC}{\bf Scalable}            
& \cellcolor[HTML]{92CDDC}{\bf Scalable}                    \\ \hline
\end{tabular*}
\captionof{table}{Structured-Light vs ToF. Source: \citep{larrylitof}}
%\caption{Dados principais do processo de metalização HVOF}
\label{tab::estructvstof}
\end{center}

\subsection{Conclusion}

As restrições apresentadas pelo problema de calibração, dentro do ambiente da
turbina, impõem um conjunto de requisitos mínimos que o sensor deve apresentar:

The sensor have some minimum requirements, imposed by the the calibration
process, that must be fulfilled. There are: 

\begin{itemize}
  \item High resolution
  \item Portability
  \item Enough reach (>20m)
  \item Withstand the temperature and humidity conditions 
\end{itemize}

In addition to these requirements, it is also desirable that the sensor has power
independent counsel and its scanning speed is not a limiting factor
for process efficiency.

The class of sensors that meets all these conditions is the Stations
measurement, except for the Nikon MV 330 sensor that does not have the portability
required for the solution, besides having a much higher price than their
competitors.

Faro Focus X330 sensor, in addition to meeting the minimum requirements, is what
has a lower price and therefore was chosen as the sensor to be used in
system calibration and accept responsibility for the spoon-spatial data environment
turbine and the robot.

However, the system specifications was not possible to guarantee perfect
sensor operation in high humidity conditions presented within the
turbine. The dipositivo operates with a system of lenses and lasers, and, if present
condensation of these components, the end result of sensing can
be impaired. Due to this fact, it performed a battery of tests on
Plant Jirau within a turbine, in order to confirm the feasibility
technique of this sensor.

% Além desses requisitos, é desejável também que o sensor tenha alimentação
% idependente e que sua velocidade de escaneamento não seja um fator limitante
% para a eficiência do processo.
% 
% A classe de sensores que atende todas essas condições é a das Estações de
% medição, com exceção do sensor Nikon MV 330 que não possui a portabilidade
% necessária para a solução proposta, além de ter um preço muito maior que de seus
% concorrentes.
% 
% O sensor Faro Focus X330, além de satisfazer os requisitos mínimos, é o que
% possui menor preço e, por isso, foi escolhido como o o sensor a ser utilizado na
% calibração do sistema e reponsável por colher os dados espacias do ambiente da
% turbina e do robô. 
% 
% Entretanto, pelas especificações do sistema não foi possível garantir a perfeita
% operação do sensor nas condições de alta umidade apresentada no interior da
% turbina. O dipositivo opera com um sistema de lentes e lasers e, caso apresente
% condensação em um desses componentes, o resultado final de sensoriamento pode
% ser prejudicado. Devido a este fato, foi realizado uma bateria de testes na
% Usina de Jirau, no interior de uma turbina, afim de confirmar a viabilidade
% técnica desse sensor.

O teste realizado constituiu na utlização de quatro esferas reflexivas,
representadas na figura \ref{fig::esferas}, distribuidas pelo ambiente da
turbina.
Em seguida, foram realizadas 4 coletas de dados, sendo 3 delas à jusante do
rotor e uma entre as pás do rotor e o distribuidor. Com a nuvem de pontos
coletada, foi possível utilizar a assinatura única gerada pelas esferas para
alinhar todos os conjuntos de dados em uma única imagem 3D.

\begin{figure}[h!]
\centering
	\includegraphics[width=0.9\columnwidth]{figs/3dsensors/kit}
	\caption{Conjunto de esferas reflexivas e tripé}
	\label{fig::esferas}
\end{figure}

A figura \ref{fig::turbina_faro} representa a vista frontal da imagem gerada e, por sua
vez, a figura \ref{fig::turbina_cad} representa uma reconstrução 3D gerada a
partir da nuvem de pontos coletas e utilizando o software proprietário do
fornecedor do sensor. 

\begin{figure}[h!]
\centering
	\includegraphics[width=0.9\columnwidth]{figs/3dsensors/recorte_video}
	\caption{Vista frontal da imagem gerada a partir dados adquiridos durante o
	teste.}
	\label{fig::turbina_faro}
\end{figure}

\begin{figure}[h!]
\centering
	\includegraphics[width=0.9\columnwidth]{figs/3dsensors/Pa_Real_Render_04}
	\caption{Reconstrução em CAD da pá com os dados do teste.}
	\label{fig::turbina_cad}
\end{figure}

A partir dos resultados gerados durante os testes foi comprovada que o sensor é
capaz de operar nas condições extremas impostas pela turbina e com um nível de
ruído aceitável para a aplicação em questão. 