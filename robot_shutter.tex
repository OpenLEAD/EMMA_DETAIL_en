
\subsection{Shutter}%TODO mudar o nome do sistema de desvio do fluxo de
% revestimento
O processo de revestimento HVOF (\textit{High Velocity Oxygen Fuel}) requer
velocidade da pistola controlada de $40~m/min$. Esta velocidade é essencial para
a qualidade do processo e deve ser mantida constante para se obter uma camada 
regular de material ao longo de toda a superfície da peça. Na solução
pesquisada demonstou-se ser inviável utilizar um robô de grande porte, devido a
limitação de acesso e ao confinamento do manipulador no ambiente. Portanto, o
manipulador  escolhido realizará o processo em regiões delimitadas da
superfície da peça e, em sua trajetória, haverá inevitavelmente mudanças de
direção, e portanto acelerações, que irão variar a velocidade da pistola.
Durante essas variações não deve-se injetar o material na peça, sendo necessário um mecanismo
autônomo para impedir o processo nestes intervalos.

A ideia inicialmente estudada foi de uma barreira (\textit{shutter}) ao fluxo na
saída da pistola.
A figura~\ref{fig::shutter_todos} ilustra a ideia para dois conceitos nas
configurações aberta e fechada. 
Nestes conceitos, uma barreira é movimentada automaticamente sempre que houver
mudança de direção da pistola, impedindo que o fluxo de material atinja a
pistola. 

\begin{figure}[h!]
   \centering
   \includegraphics[width=0.8\columnwidth]{figs/shutter/shutter_todos}
   \caption{Conceitos de \textit{shutter} avaliados}
   \label{fig::shutter_todos}
\end{figure}

Algumas considerações foram levantadas para avaliar a viabilidade
desta solução, como a alta temperatura da chama, a capacidade do atuador, a
resistência mecânica da barreira e a taxa de acúmulo de material. Este conceito
foi abandonado principalmente devido ao acúmulo de material na barreira, o que
levaria a um aumento de seu peso, e por consequência momento de inércia,
alterando a dinâmica prevista, ou ainda, poderia chegar a obstruir a saída da
chama causando danos à pistola.

Outra proposta que está sendo estudada é a de modificar o fluxo da linha de
revestimento. A ideia é a inclusão de uma válvula direcional com atuação por
solenóide para desviar o fluxo do material de revestimento para um tanque ou
cilindro de retorno. Esta atuação deve ser autônoma e coordenada com a
trajetória do manipulador. A válvula seria de três vias e duas posições ($3/2$) tal que, no 
repouso, direciona-se o fluxo diretamente para a pistola e, quando
atuada, bloqueia-se o fluxo para a pistola e abre-se o fluxo para exaustão. Uma
válvula limitadora de pressão regulável seria utilizada na linha de exaustão
para igualar as diferenças de pressão entre as duas vias, minimizando efeitos
transitórios.
Outra característica opcional importante para redução dos efeitos transitórios,
como pico de pressão, é a de sobreposição aberta, ou seja, o fluxo só seria
fechado da posição inicial quando o movimento de troca estivesse completo. 
A figura~\ref{fig::circuito_hvof} apresenta o circuito do processo HVOF de forma
simplificada.

 \begin{figure}[h!]
   \centering
   \includegraphics[width=0.8\columnwidth]{figs/shutter/Circuito_HVOF_mod}
   \caption{Circuito do processo HVOF modificado}
   \label{fig::circuito_hvof}
\end{figure}

A linha tracejada representa o circuito original, as linhas em vermelho
representam a modificação do circuito com os equipamentos adicionais indicados.

Esta é uma alternativa que tem como principal vantagem a de poder retornar a
matéria-prima do revestimento para tanque, ou seja, evita-se
o desperdício do material no ambiente. Esta matéria-prima poderia então ser
reaproveitada no processo, separando-se o gás.
