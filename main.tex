\documentclass{main}
    

\usepackage{graphicx}      % include this line if your document contains figures
\usepackage{natbib}        % required for bibliography
\usepackage{enumerate}
\usepackage[utf8]{inputenc}
\usepackage{float}
\usepackage[centerlast,small,sc]{caption} 
\setlength{\captionmargin}{30pt}
\usepackage[none]{hyphenat} 
\usepackage{subcaption} 
\usepackage{capt-of}
\usepackage{kpfonts}
\usepackage{calc}
\newlength\shlength
\newcommand\xshlongvec[2][0]{\setlength\shlength{#1pt}%
  \stackengine{-5.6pt}{$#2$}{\smash{$\kern\shlength%
    \stackengine{7.55pt}{$\mathchar"017E$}%
      {\rule{\widthof{$#2$}}{.57pt}\kern.4pt}{O}{r}{F}{F}{L}\kern-\shlength$}}%
      {O}{c}{F}{T}{S}}
\sloppy  

\begin{document}

\begin{frontmatter}

\title{State of the art of robotic solutions for \textit{in situ} hard coating
of hydraulic turbines\thanksref{footnoteinfo}} 

\thanks[footnoteinfo]{This work is supported by ESBR under contract COPPETEC
JIRAU 09/15 6631-0003/2015 (ANEEL R\&D program).}

\author[1]{Renan S. Freitas}
\author[1]{Gabriel Alcantara C. S.}
\author[1]{Eduardo Elael M. S.}
\author[1]{Estevão Fróes}
\author[1]{Ramon R. Costa}

\address[1]{Department of Electrical Engineering, COPPE UFRJ, Rio de Janeiro, Brasil} 
  
\begin{abstract}                % Abstract of not more than 250 words.
%TODO Renan: Resumo
\end{abstract} 
 
\begin{keyword}
%TODO Renan: Keywords
\end{keyword}

\end{frontmatter}

\section{Introduction}

%%%%%%%%%%%%

The importance of regular maintenance on hydraulic turbines have already been
established (ref EMMA-SOTA), given that its power output have a near 46\%
increase after maintenance. Providing a meaningful gain for highly dependent
countries like Brasil and Norway.

% Em EMMA-SOTA, é apresentada a importância da manutenção regular das turbinas em
% uma usina hidrelétrica, já que, em sua operação ideal e de máxima eficiência,
% sua potência tem aumento de quase 46\% após manutenção. Aumento significativo,
% principalmente para países dependentes desta forma de energia, como o Brasil e
% Noruega.

The efficiency of a hydraulic turbine is related at some extend to the hydraulic
profile of the runner blades and its degradation is mainly due to two
phenomena: cavitation and abrasion. As a protective measure, the blades are
hardcoated by a process called HVOF (High velocity oxy-fuel coating spraying),
which mitigate the damage caused by the aforementioned effects, but has to be
reapplied periodically.

The hardcoating process can take up to two months per turbine, including turbine
disassemble, blade's hardcoating on a specifically designed environment and
following remounting and calibration.

% A eficiência de uma turbina hidrelétrica depende de inúmeras variáveis, como
% volume de água, queda d'água, o tipo da turbina, o distribuidor e outras. O
% projeto EMMA tem foco na manutenção do perfil hidráulico das pás dos rotores de
% turbinas hidrelétricas, por este se degradar com maior rapidez, exigindo
% manutenções recorrentes. 

% A fim de proteger a pá contra abrasão e cavitação é realizado processo de
% revestimento por asperção térmica, ou, especificamente, a metalização (HVOF).
% Atualmente, este processo pode levar cerca de dois meses por turbina,
% já que exige que a turbina seja desmontada, as pás serem processadas em
% outro ambiente, a turbina seja remontada e recalibrada.

Aiming to reduce the downtime assossiated for the hardcoating process, robotic
\textit{in situ} solutions, i.e. inside the runner environemnt, consisting of a
insdustrial robotic arm mounted over a customized base are scrutinized. This is
the main objetive of the EMMA project, a R\&D project by Fundação Coordenação
de Projetos, Pesquisas e Estudos Tecnológicos (COPPETEC), in partnership with
Rijeza company, Agência Nacional de Energia Elétrica (ANEEL) and Energia
Sustentável do Brasil (ESBR).

Despite the generic approach of this article, the bulb turbine facilities are
different for each power plant. The reference hydroeletric power plant for the
ideas discussed here is the UHE Jirau, near Porto Velho(Brazil - RO). The
Madeira river, where the UHE Jirau is located, has a high concentration of
suspended particles which entails further abrasion on the blades, it also has a
low water level diference of 2 to 20 meter intensifing cavitation. For more
details on the UHE Jirau please reference (EMMA SOTA), however it is important
to recall that there are two main entry points for runner area: the top hatch
with a 35.7 cm diameter, just above the turbine, and the bottom hatch with a 80
cm diameter on the draft tube, 10 m away from the runner and with a 4m access
duct from exterior ground level. But, as a top hatch is not a standard across
other hydroeletric power plants, e.g. the UHE de Santo Antônio near UHE Jirau,
the solutions focus on the bottom hatch.

The workflow for the \textit{in situ} hardcoating can be thought as a
sequence of 4 minor jobs:
Enter the runner area with the robot; Move the robot and anchor it in some
suitable positions near the blade to be hardcoated; Calibrate the robot, in the
sense of identifying the relative positions of the robot, blade and the rest of
the environment; And finally, check if the the robot's arm can cover the whole
blade.

The following sections explore the developments on the EMMA project, they are
organized so to mimic the workflow:
section 2 exposes the ideas for the customized base and related logistics;
section 3 describes the calibration process which has to be done once the robot
is well placed, but before it's able to start coating; section 4 explore the
robot arm's limitations for performing the hardcoating; And lastly section 5
concludes and discusses the future steps for the EMMA project.

% Apesar de o projeto visar uma solução genérica para turbinas bulbo, as
% instalações são diferentes em cada usina. Desta forma, o ambiente de testes
% deste projeto é a Usina Hidrelétrica de Jirau, localizada no Rio Madeira. O Rio
% Madeira carrega muitos sedimentos provocando maior abrasão nas pás, se comparado
% com outros usinas, além disso, a queda d'água de 2 a 20 metros intensifica o
% fenômeno de cavitação. As principais características das instalações da turbina
% em análise estão descritas em EMMA-SOTA, mas vale ressaltar a particularidade
% dos dois acessos principais ao aro câmara, relevantes para a busca de uma
% solução: acesso superior (35.7 cm de diâmetro) e acesso inferior (80 cm de
% diâmetro).

% O projeto EMMA busca uma solução para o processo de metalização \textit{in
% situ}, isto é, revestimento das pás no ambiente da turbina, diminuindo o tempo
% de manutenção e, consequentemente, de máquina parada.  A solução conceitual
% desenvolvida em EMMA-SOTA é a utilização de um manipulador industrial sobre uma
% base. As características do manipulador e da base variam de acordo com o
% acesso: no caso da escotilha superior, a solução é um manipulador industrial de
% pequeno porte e base customizada operada eletronicamente; no caso da escotilha
% inferior, a solução é um manipulador industrial de porte médio e base tipo
% trilho com acopladores magnéticos.
% 
% A análise das instalações da Usina Hidrelétrica de Santo Antônio, em Porto
% Velho, vizinha à Jirau, mostrou que as turbinas não possuem um acesso superior.
% A fim de tentar construir uma solução mais geral, o presente documento visa dar
% continuidade ao projeto, detalhando o estudo de viabilidade técnica para a
% solução da escotilha inferior.

%%%%%%%%%%%%%%%%%%%%%%%%%%%%%%%%%%%%%%%%%%%%%%%%%%%%%%%%%%%%%%%%%%%%%%%%%%%%
% According to the world energy council, hydropower is the most flexible and
% consistent of the renewable energy resources and, at the end of 2008, the total
% capacity of hydropower resources was 874 GW. Brazil is the second
% country in hydropower production, and second with the highest
% consumption of hydropower with a 70.000 MW installed capacity, and 433
% hydroelectric plants in operation. Since Brazil is one of the world's richest
% countries in water resources, and the hydropower is the most dominant across
% the country, it motivates the development and investment in hydropower
% generation., and Brazil is the .
% %\begin{figure}[h!]
% %	\includegraphics[width=\columnwidth]{figs/intro/graph.png}
% %	\caption{Top hydropower producing countries}
% %	\label{fig::cavitacao}
% %\end{figure}
% 
% %O Brasil é um dos países mais ricos do mundo em recursos hídricos, facilitando
% % o desenvolvimento e investimento em geração de energia a partir desse recurso. A
% %energia hidráulica é a mais dominante em todo o país, e o Brasil é o segundo
% %país com maior consumo de energia hidrelétrica no mundo com capacidade
% %instalada de 70.000 MW, 433 usinas hidrelétricas em operação. 
% 
% In Brazil, the renovation and improvement of the built large plants is estimated
% to result in a potential increase of 32.000 MW \citep{goldemberg2007energia}, a
% figure that can be achieved, in large part, by the maintenance of the
% hydropower turbines. These turbines are constantly exposed to abrasion and
% cavitation phenomena, which determine its life cycle.
% %Estima-se que a reforma e melhoria das grandes usinas construídas resultariam
% %em um aumento potencial de 32.000 MW \citep{goldemberg2007energia},
% %número que pode ser alcançado, em grande parte, pela manutenção das turbinas
% %geradoras da energia elétrica. As turbinas estão constantemente expostas aos
% %fenômenos de abrasão e cavitação, os quais determinam sua vida útil.
% 
% The cavitation phenomenon is very well studied and detailed in
% \cite{escaler2006detection}, which outlines their types, occurrences and
% effects in the different hydraulic turbines. This physical phenomenon can cause
% erosions in the hydraulic turbines, leading to water flow instability,
% excessive vibrations and turbine efficiency reduction.
% %O fenômeno de cavitação está muito bem estudado e detalhado em
% %\cite{escaler2006detection}, onde são apresentadas seus tipos, ocorrências e os
% %efeitos nas diferentes turbinas. Esse fenômeno físico pode causar erosões na
% %máquina hidráulica (figura~\ref{fig::cavitacao}), gerando instabilidade de
% % fluxo de água, vibrações excessivas e redução da eficiência da turbina.
% 
% \begin{figure}[h!]	
% 	\includegraphics[width=\columnwidth]{figs/intro/cavitacao2}
% 	\caption{Jirau hydraulic turbine blade eroded by cavitation.}
% 	\label{fig::cavitacao}
% \end{figure}
% 
% Hard coating techniques by thermal aspersion
% are used to reduce the erosion of the turbine's blade from cavitation or
% abrasion, thus increasing its life cycle. This solution is analogous to a paint
% that protects walls from environment exposure. The hard coating procedure is performed
% before the hydraulic turbine installation by a robotic manipulator. The
% procedure requires a robotic system due to high precision, speed, and
% the hazardous substances that are used, as propane and other gases.
% Although sufficient for blade protection, the coating also has a life
% cycle itself, thus it needs to be redone from time to time to ensure the
% blade's protection from physical phenomena.
% %A fim de reduzir o desgaste da pá contra cavitação ou abrasão e aumentar a sua
% %vida útil, utiliza-se a técnica de revestimento por asperção térmica, que pode
% %% ser comparada com uma tinta que protege à exposição com o ambiente. O
% % procedimento é realizado
% %antes da instalação das pás na turbina por um robô, pois exige alta precisão
% %e velocidade, além de expelir substâncias nocivas à saúde. Apesar de suficiente
% % para a proteção da pá, o revestimento também tem vida útil e precisa ser refeito de tempos em tempos para
% %garantir a proteção da pá contra os fenômenos físicos.
% 
% In the specific case of the Jirau hydroelectric dam, built on the Madeira
% river, the number of suspended particles that the river carries intensifies the
% abrasion phenomena, and Rijeza, a hard coating specialized company, identified cavitation erosion on blades, further reducing the coating life cycle.
% Therefore, Jirau hydroelectric dam needs regular maintenance, which,
% in the present situation, would require stoppage of the turbine, removing the
% blades, positioning the blades for coating, coating application, turbine assembling, and recalibration. The downtime to perform all
% maintenance can take up to two months, meaning a huge loss in power generation .
% %No caso específico da usina hidrelétrica de Jirau, construída no rio Madeira,
% %os fenômenos de abrasão são intensos devido ao grande número
% %de partículas que o rio carrega diariamente, reduzindo ainda mais a vida útil
% % do revestimento.
% %Portanto, há a necessidade de manutenção regular, o que, na situação atual,
% %exige paralização da máquina, desmontagem da turbina, posicionamento de cada pá
% %na área designada ao revestimento, aplicação do revestimento, montagem da
% %turbina e recalibração. O tempo de paralização para a realização de
% %toda a manutenção pode levar de um a dois meses, significando uma grande perda
% %na geração de energia. 
% 
% EMMA is an R\&D project by Fundação Coordenação de Projetos, Pesquisas e Estudos
% Tecnológicos (COPPETEC), in partnership with Rijeza company, Agência Nacional de
% Energia Elétrica (ANEEL) and Energia Sustentável do Brasil (ESBR). Its first
% stage is a technical feasibility study of a robotic system to perform
% coating by thermal spray on hydraulic turbine blades within the turbine
% environment. The project aims to significantly reduce the downtime for hard
% coating process.
% %A primeira etapa do projeto EMMA, pesquisa e desenvolvimento
% %realizados pela Fundação COPPETEC, em parceria com a empresa Rijeza, ANEEL e
% %ESBR, é um estudo de viabilidade técnica de um sistema robótico para realizar
% %revestimento por aspersão térmica de turbinas \textit{in situ}, ou seja, dentro
% %do ambiente da turbina (aro câmara). O projeto tem como objetivo reduzir
% %significativamente o tempo de manutenção do revestimento por ser realizado no
% %ambiente confinado da turbina e, portanto, não havendo necessidade de sua
% %desmontagem.
% 
% This document is divided as follows: section 2 describes, in detail, the
% problem, contextualizes the reader in the Jirau environment and
% describes the robot's tasks; section 3 surveys the state of the
% art; section 4 describes the conceptual designs for the robot and mechanical
% bases; finally, the section 5 concludes and outlines the next steps for the
% EMMA project.
% %Este documento está dividido da seguinte forma: a seção 2 descreve
% %detalhadamente o problema, contextualiza o leitor no ambiente da usina de
% %Jirau e descreve as possíveis tarefas do robô; a seção 3 faz um levantamento do
% %estado da arte; a seção 4 descreve os projetos conceituais para o robô; e a
% %seção 5 conclui e descreve os próximos passos para o projeto EMMA. 
 
\section{The problem}\label{sec::consideracoes}

Cavitation and abrasion in hydropower turbines cause surface erosion and
blade's hydraulic profile deformation, resulting in efficiency
reduction. A preventive solution is the High Velocity Oxygen Fuel (HVOF) coating
process of the blades. The hard coating creates a more lamellar structure,
increasing power generation efficiency, and provides greater
resistance to erosion. In the case of the Jirau hydroelectric dam, the coating
of turbine's blades is performed before turbine assembling and
installation. However, the abrasion due to a large number
of particles and sediment in the Madeira river and the recent identified
cavitation require recoating in short intervals \citep{santa2009slurry}. Turbine
disassembling, recoating and turbine reassembling are very expensive process
and should be done regularly.

%O fenômeno de cavitação e abrasão em hidroturbinas provoca desgaste
%superficial por erosão e alteração do perfil
%hidráulico da pá, gerando redução da eficiência na geração de energia.
%Uma solução preventiva é o revestimento por metalização das pás, o qual aumenta
% a eficiência na geração de energia por gerar uma estrutura mais lamelar, e fornece maior
%resistência a desgastes. No caso da usina hidrelétrica de Jirau, o revestimento
%das pás é realizado antes da montagem e instalação da turbina, porém devido ao
% grande número de partículas e sedimentos que o rio madeira carrega e à cavitação, o revestimento
%deve ser aplicado novamente em intervalos curtos de tempo
%\citep{santa2009slurry}. A desmontagem da turbina, aplicação de novo
%revestimento nas pás e remontagem são um processo muito custoso e deverá ser
%feito regularmente. Portanto, há a necessidade de o procedimento ser
%executado dentro do aro câmara, \textit{in situ}, onde as pás são instaladas.

Cavitation is the formation of vapor cavities (bubbles) in a liquid due to
sudden pressure drops. When the liquid is subjected to increased pressure,
the bubbles implode, causing shock waves \citep{brennen2013cavitation}. 
In hydropower turbines, the cavitation occurs near the blades or
in the turbine output.% The liquid has the combination of kinetic components,
%gravitational potential, and energy flow. The kinetic component is due to the
%water flow (velocity), the potential depends on the altitude of the liquid, and
%the energy flow is the energy that a fluid contains due to pressure. According
%to the Bernoulli principle, the conservation of fluids, it implies that for
%the same altitude, increased kinetic component causes a reduction in pressure,
%occurring cavitation. 
The formation of large bubbles modifies the
characteristics of the flow, causing oscillations or vibrations on the
hydropower turbine, and negative affects the performance of the hydraulic
system. On the other hand, the collapse of small bubbles generates high
frequency shock waves, causing erosions on metal surface.
%A cavitação é a formação de cavidades de vapor (bolhas), em um líquido, devido
% a quedas repentinas de pressão. Quando o líquido é sujeito ao aumento de pressão,
%as bolhas implodem, ocasionando ondas de choque \citep{brennen2013cavitation}.
%Em hidroturbinas, o fenômeno de cavitação é comum próximo às pás ou
%na saída da turbina. O líquido apresenta a combinação
%de componentes cinético, potencial gravitacional e energia de fluxo. O
%componente cinético é em virtude do fluxo da água (velocidade), o potencial tem
%relação com a altitude, e a energia de fluxo é energia que um fluido contém
%devido à pressão que possui. De acordo com o princípio de Bernoulli, o
% princípio da conservação para os fluidos, implica-se que, para uma mesma altitude, o
%aumento da componente cinética acarreta em uma diminuição da pressão, ocorrendo
%cavitação. 
%Quando há cavitação, a formação de bolhas grandes altera as características do
%escoamento, ocasionando oscilações ou vibrações na máquina que, por
%conseqüência, prejudicam o rendimento do sistema hidráulico. As bolhas
%pequenas, ao colapsar, geram ondas de choque de alta frequência, podendo
% provocar erosões se próximo à superfície metálica.

%Além da cavitação, como a água atravessa o aro câmara em grande velocidade, o
%acúmulo de sedimentos irá provocar desgaste abrasivo, isto é, perda de material
%pela passagem de párticulas rígidas. 

In this section, the coating technology is addressed as the way to
reduce the damage by cavitation. Also, the reader is contextualized in
the Jirau hydroelectric plant problem. Finally,  robotic tasks are highlighted to
solve the problem.
%Nesta seção, são apresentadas as formas de reduzir os danos da cavitação pela
%tecnologia de revestimento por metalização, a contextualização do problema no
%caso da usina hidrelétrica de Jirau e as tarefas que um sistema robótico deve
%realizar para solucionar o problema.


\subsection{The HVOF coating process}\label{sec::desc_hvof}
The thermal spraying (or metallizing) is a process in which
heated materials are sprayed onto a surface in order to improve or
restore their properties and dimensions. The coating extends the material life
cycle, significantly increasing its resistance to erosion and corrosion. 
%The different types of thermal spraying are: plasma spraying, detonation
%spraying, wire arc spraying, flame spraying, High velocity oxy-fuel coating
%spraying (HVOF), warm spraying and cold spraying.

%O revestimento por aspersão térmica (ou metalização) é um processo em que
%partículas aquecidas são pulverizadas em uma superfície a fim de melhorar ou
%restaurar suas propriedades e dimensões. O revestimento estende a vida útil do
%material, aumentando significantemente a sua resistência à erosão e corrosão.
%Os diferentes tipos de metalização são: por chama, arco elétrico, detonação,
%chama de alta velocidade (HVOF), plasma, a frio e a quente.

A thermal spraying system comprises: a spray gun, responsible
by melting and accelerating the particles to be deposited onto metal
surface; a feeder, which provides the powder (particles) via pipes;
a provider of burning material; a robot (manipulator) to handle the gun; an
electric power supply to the gun; and a control console for the system.
%Um sistema de metalização é composto por: uma pistola de aspersão, responsável
%pelo derretimento e aceleração das partículas a serem depositadas na
%superfície; um alimentador, que fornece o pó (partículas) através de tubos;
%um fornecedor do material de combustão; um robô para manipular a pistola; uma
%fonte de alimentação elétrica para a pistola; um console de controle para o
%sistema.

In the specific case of the Jirau, turbine blades (stainless steel 420)%,
%the thermal spraying type is HVOF on both sides of the blade, and executed 
are coated on High Velocity Oxy-Fuel (HVOF) thermal spraying type by the
Rijeza company with a robotic manipulator 150 kg payload, which is a good
safety margin, since the system mass is 10 kg (cables and gun). The process
takes 6 hours per side of the blade.
%No caso específico das pás (aço inox 420) das turbinas da usina hidrelétrica de
%Jirau, antes da montagem da turbina, a metalização tipo HVOF é realizada em
%ambos os lados da pá pela empresa Rijeza com um manipulador industrial de 150
% kg de carga máxima, permitindo controle de vibrações com boa margem de segurança, já que a massa do
%sistema pode chegar a 10 kg (cabos e pistola). O tempo
%mínimo do processo é de 6 horas por lado da pá.

The HVOF consists of feeding, in a combustion chamber, the coating material
(tungsten carbide), a gaseous fuel mixture (propane), and oxygen. According to
the data provided by the Rijeza company, the 8 kg spray gun projects a flame of
$3000^oC$, spraying the particles with 700 to 1000 m/s speed, generating a
15 N recoil force.
%O HVOF consiste em alimentar, numa câmara de combustão, o material de
%revestimento (carboneto de tungstênio), uma mistura gasosa do combustível
% (propano) e oxigênio. De acordo com os dados fornecidos pela empresa Rijeza, a pistola de 8
%Kg projeta uma chama de $3000^oC$, que pulveriza as partículas com velocidade
% de 700 a 1000 m/s, gerando uma força de recuo de 15 N.

The robotic manipulator must have 5 mm accuracy, and the spray gun should remain
at a 230 to 240 mm distance with $90^o\pm 60^o$ angle in respect to the metal
surface plane. The end effector of the manipulator must control the spray
gun at a constant 40 m/min speed and should not stop during the process in a blade range
position (\textit{long stop}), otherwise coating material would accumulate.
The end effector direction changes are considered \textit{long stop} too, thus
direction changes should be made out of the blade range, or sacrifice plates
should be used. Sacrifice plates, or masking, are metal plates placed on blade
spots where should not be coated. It is, usually, a plate of common steel as the
flame from the spray gun does not stand still for a long period, not heating it
enough to damage. 
%O manipulador robótico deve possuir precisão de 5 mm, a pistola no efetuador
%deve permanecer a uma distância que varia entre 230 e 240 mm, e ângulo de $90^o
%\pm 60^o$, em relação à superfície. O manipulador deve ser capaz de
%mover a pistola a velocidade constante de 40 m/min, e não pode permanecer uma
%posição da pá por muito tempo (parada), pois há acúmulo de material, deformando
% a superfície. Trocas de direção ou sentido na movimentação do manipulador são
%considerados como parada, logo as trocas deverão ser realizadas em áreas
%exteriores à superfície da pá ou chapas de sacrifício são utilizadas. 
%Placas de sacrifício, ou mascaramento, são chapas colocadas em regões onde as
%peça não podem ser jateadas ou revestidas. Geralmente uma chapa de qualquer
% tipo de aço pode ser utilizada, pois a chama não fica parada sobre ela por um longo
%período, não aquecendo-a o suficiente para danificar. Quando a pistola
%permanece, em funcionamento, a chama é apontada para algum lugar onde não tenha
% obstáculos.

 % As informações do processo
% podem ser observadas na figura~\ref{fig::hvof}.
 
%\begin{figure}[h!]	
%	\includegraphics[width=\columnwidth]{figs/intro/hvof.pdf}
%	\caption{Foto do efetuador do manipulador e pistola HVOF.}
%	\label{fig::hvof}
%\end{figure}

Regarding the operating conditions, the hydropower turbine is a confined
space, the HVOF process has excessive audible noise (100-140 dB), harmful gases
and potentially explosive gases are expelled; the blade can reach up to
$110^oC$; the environment temperature and humidity should be monitored and
ideally set for the coating process; and $40\%$ of the sprayed particles are
lost during the process \citep{wu2006rebound}, which are spread throughout
the environment. Therefore, some measures must be taken for the execution: the
operation must be teleoperated; no human presence; environment gases, humidity
and temperature must be constantly monitored; the robotic manipulator should be
sealed; %the wasted particles should be removed afer process (cleaning); 
and system electric/electronic shutdown must be accompanied by gas cutting.
%Em relação às condições de operação: o espaço da aplicação HVOF é confinado,
%com excesso (100 a 140 dB), gases nocivos e com risco de explosão podem
%ser exalados; a pá pode atingir temperaturas de até $110^oC$; as condições de
%umidade e temperatura devem ser ideais para o processo; e há perda de $40\%$
%das partículas pulverizadas  \citep{wu2006rebound}, que são espalhadas pelo
%ambiente. Portanto, algumas medidas devem ser tomadas para a execução do
%processo: a operação deve ser remota, não há presença de pessoas \textit{in
%loco}; os gases presentes e umidade/temperatura devem ser constantemente
%monitorados; o robô manipulador é selado; as partículas desperdiçadas devem
%ser removidas (limpeza); e o desligamento do sistema deve ser acompanhado por
% corte de gás.

%Finally, the coating quality is evaluated by an instrument which measures
%surface porosity, oxidation , hardness and roughness. The measurement is
%performed manually, quickly and easily by an operator.
%A qualidade do revestimento é geralmente avaliada por um instrumento que
%realiza a medida de porosidade, oxidação, dureza e rugosidade da superfície. O
%processo é realizado manualmente, de maneira rápida e fácil, por um operador.


The table~\ref{tab::hvof} summarizes the project restrictions and specifications:
%A tabela~\ref{tab::hvof} resume as restrições e especificações do
%projeto:

\begin{center}
\begin{tabular}{  c | c  }
  \hline
  \textbf{Component} & \textbf{Data} \\ \hline
  Spray gun mass & 8 Kg  \\ \hline
  Cables mass & 12 Kg  \\ \hline
  Processing time per blade & 12 horas \\ \hline
  Flame temperature & $3000^oC$ \\ \hline
  Spray gun recoil & 15 N \\ \hline
  Manipulator precision & 5 mm \\ \hline
  Spray gun to blade distance & 230-240 mm \\ \hline
  Spray gun to blade angle & $30^o$-$90^o$ \\ \hline
  Manipulator velocity & 40 m/s \\ \hline
  HVOF sound noise & 100 a 140 dB \\ \hline
  Blade temperature & $110^oC$ \\
  \hline
\end{tabular}
\captionof{table}{HVOF process data}
%\caption{Dados principais do processo de metalização HVOF}
\label{tab::hvof}
\end{center}


%\begin{center}
%\begin{tabular}{  c | c  }
%  \hline
%  \textbf{Componente} & \textbf{Dado} \\ \hline
%  Massa da pistola HVOF & 8 Kg  \\ \hline
%  Massa dos cabos HVOF & 12 Kg  \\ \hline
%  Tempo HVOF por pá & 6 horas \\ \hline
%  Temperatura da chama HVOF & $3000^oC$ \\ \hline
%  Recuo da pistola & 15 N \\ \hline
%  Precisão do manipulador & 5 mm \\ \hline
%  Distância pistola-pá & 230-240 mm \\ \hline
%  Ângulo pistola-pá & $30^o$-$90^o$ \\ \hline
%  Velocidade do manipulador & 40 m/s \\ \hline
%  Ruído HVOF & 100 a 140 dB \\ \hline
%  Temperatura da pá & $110^oC$ \\
%  \hline
%\end{tabular}
%\captionof{table}{Dados principais do processo HVOF}
%\caption{Dados principais do processo de metalização HVOF}
%\label{tab::hvof}
%\end{center}

%Sistemas robóticos não devem utilizar magnetismo como meio de aderência, já que
%o aço inox 420 não apresenta alta permeabilidade magnética e a alta temperatura
%da pá deve inviabilizar essa solução. Adesão por ventosas é uma solução
%viável, pois material não causa dano ao revestimento, porém a escolha do
%material da ventosa deve ser estudado,já que a pá quente pode ocasionar em
%perda de sucção, como em ventosas emborrachadas.


\subsection{Requirements for HVOF process}

The HVOF process in hydropower turbines has some prerequisites for correct
coating application: sandblasting, and repare. %This subsection details the
%appropriate blades surfaces preparation to ensure the maintenance of the
% hydraulic profile.
%O processo de metalização de turbinas hidrelétricas tem alguns pré-requisitos
%que devem ser respeitados para uma correta aplicação e fixação da camada de
%material durante o revestimento. Essa subseção descreverá as etapas necessárias
%de preparação da superfície a fim de se assegurar a manutenção da qualidade dos
% resultados e do perfil hidráulico da pá. 

%\subsubsection{Sandblasting}\label{sandblasting}

According to Rijeza company studies, recoating a coating surface by
superimposition does not produce satisfactory results, thus it is recommended
a abrasive blasting operation to smooth the surface and remove the remaining
coating. Sandblasting consists in forcibly propelling a stream of abrasive material
against a surface under high pressure. The erosion creates an uniform
superficial layer, increasing the surface roughness and
adhesion.
%O processo de metalização sobreposto a uma superfície que já possui uma camada
%protetora desgastada não apresenta um resultado tão satisfatório se comparado
%com o processo realizado em uma superfície crua. Por esse motivo é recomendado
%que seja realizado um processo de jateamento abrasivo. 

%O jateamento consiste em direcionar um fluxo de material abrasivo na superfície
%do material a fim de se erodir a mesma e retirar o material depositado na
% camada superficial. Outra característica desse processo é a capacidade de aumentar a
%rugosidade da superfície e, assim, aumentar o poder de adesão da nova camada a
%ser metalizada.  

%In the specific case of turbine blades, the sandblasting process utilizes
%aluminum oxide as abrasive material. An operator performs the sandblasting with
%specialized equipment, which is composed a compressed air, and particles
%of the abrasive jet.
%O processo de jateamento para o tratamento específico da superfície das pás da
%turbina utiliza óxido de alumínio como material abrasivo e pode ser realizado
%por um operador. A infraestrutura necessária para esse processo é uma fonte de
%ar comprimido, geralmente proveniente de um compressor de ar, para propulsionar
%o particulado que forma o jato abrasivo. %\textbf{A preparação do ambiente no
% envolto da pá, o escoamento do material e
%as consequências da realização desse processo não foram analisados} e,
%possivelmente, será necessária a implementação de infraestrutura de suporte
%para proteção dos equipamentos adjacentes que não receberão o jateamento,
%limpeza do material depositado e exaustão do particulado suspenso.

%\subsubsection{Repare}

Damages on the blade surface can reduce its efficiency and even its integrity,
committing operation safety. The HVOF process can not repair severe surface
damage or structural damage, such as cracks. Therefore, blade inspection should
be performed before HVOF process, since the sandblasted surface facilitates
damage visualization.
%Danos existentes na superfície da pá ou em sua estrutura podem reduzir a sua
%eficiência e até mesmo a própria integridade da pá, prejudicando a segurança 
%da operação. O processo de metalização não tem a capacidade de reparar
%danos severos na superfície ou danos estruturais como rachaduras. A inspeção
%para procura desses defeitos deve ser realizada antes da realização do processo
%de metalização, uma vez que a superfície jateada, ou seja, em metal cru sem
%camada de proteção, facilita a visualização de danos. %Os procedimentos para
%reparos de danos estruturais ou referentes a rachaduras não serão cobertos por
%este documento.
%The damages caused by cavitation, as explained in
%section~\ref{sec::consideracoes}, modifies the blade hydraulic profile and
%should be repaired. 
The repair procedure varies according to the severity of
the damage. Common repair procedures are: 1) Non fused materials; 2) Welding;  
3) Welding and solid plate.
%Os danos causado por cavitação, como explicado na seção
%\ref{sec::consideracoes}, pode alterar o perfil hidráulico da pá e deve ser
%reparado sempre que possível. O procedimento de reparo varia de acordo com a
%severidade dos danos causados. À medida que a profundidade das cavidades
% geradas na pá e a extensão dos danos vão aumentando, medidas mais extremas se tornam
%necessárias e, por isso, a estratégia de reparo para esse tipo de dano deve
%estar alinhada com o tipo de processo que se deseja utilizar. Inspeções e
%reparos mais frequentes significam processos mais simples, enquanto que reparos
%mais espaçados podem resultar até na inutilização da pá. Os procedimentos mais
%utilizados para o reparo de danos causados por cavitação são:

%\begin{itemize}
%  \item Non fused materials;
%  \item Welding;
%  \item Welding and solid plate.
%\end{itemize}

%\paragraph{Repair with non fused materials}

%Minor damages are usually treated with non fused material processes, where it
% is not necessary to fuse materials to fill the blade cavities. The processes and
%materials used are: epoxy, ceramics, HVOF coating, neoprene and urethane.
%Para pequenos danos, é possível utilizar processos nos quais não é necessário
%fundir o material depositado para preenchimentos das cavidades ao material da
%superfície metálica da pá. Os processos e materiais utilizados, usualmente na
%indústria, são: 

%\begin{itemize}
%\item Epoxy;
%\item Cerâmica;
%\item Revestimento por metalização;
%\item Neoprene;
%\item Urethane.
%\end{itemize}

%Vale ressaltar que a solução proposta para a metalização de uma camada
% protetora para se evitar os danos causados pela cavitação também poderia ser utilizado
%para preencher danos passados, desde que respeitem o limite de espessura
%para o tipo de processo utilizado


%\paragraph{Repair with welding}

%Cavitation repair by welding is the most common procedure, as it allows high
%material deposition and requires less maintenance. This process consists in
%welding layers deposition. After the operation, the repairing surface should be
%polished accordingly with the standard measures of quality for the blade
%hydraulic profile. This procedure is usually performed manually by a highly
%skilled operator. There are, also, in the literature, automated solutions, such
%as the Scompi Robot and Roboturb \citep{roboturb,scompi}.
%O preenchimento dos danos causados devido à cavitação por solda é o
%procedimento mais comum, pois possibilita uma maior deposição de material e não
% obriga a realização de reparos com uma frequência elevada. Esse processo consiste na
%deposição de solda em camadas, até o completo preenchimento. A
%superfície deve ser, então, esmerilhada até entrar em conformidade com as
%medidas padrão de qualidade para o perfil hidráulico da pá a ser reparada. Essa
% tarefa é normalmente realizada por mão de obra altamente qualificada e existe, também,
%na literatura a presença de soluções automatizadas, como os robôs Roboturb e
% Scompi
%\citep{roboturb,scompi}

%\paragraph{Repair by welding and solid plate}

%Severe damages are usually repaired by solid plates, which would fill large
%areas. The fixation process of the plates is carried out by welding, as well
%as filling the remaining cavities. 
%Para casos de danos mais severos, pode ser necessária a utilização de placas
%para o preenchimento de grandes extensões. O processo de fixação das placas é
%realizado por solda, assim como o preenchimento do volume restante. O processo
%de solda, esmerilhamento e verificação é comum ao procedimento padrão
% utilizando somente solda.
\subsection{Environment contextualization}\label{sec::desc_contex}

The Jirau plant has horizontal axis bulb type turbines. In hydropower
plants, electric power generation depends on the water level and river flow,
however bulb type turbines are designed just for high water flow to produce
enough electricity. %The figure\ref{fig::bulb_turbine} and the
%table\ref{tab::bulb_turbine} illustrate a bulb type turbine and the large
%ducts needed to hold the large volume of water.
%A usina hidrelétrica de Jirau é do tipo fio d'água, na qual são utilizadas
% turbinas do tipo bulbo de eixo horizontal. Como a geração de energia depende da altura da queda d'água e da vazão do rio, as turbinas do tipo bulbo utilizam uma grande vazão de
%água para produzirem energia elétrica suficiente. A figura
%\ref{fig::bulb_turbine} e a tabela \ref{tab::bulb_turbine} ilustram uma turbina
%do tipo bulbo e o grandes dutos necessários para comportar o grande volume de
% água que passa através da turbina.
 
%\begin{figure}[h!]	
%	\includegraphics[width=\columnwidth]{figs/intro/bulb_turbine2}
%	\caption{Bulb type turbine.}
%	\label{fig::bulb_turbine}
%\end{figure}

%\begin{center}
%\begin{tabular}{  c | c  }
%  \hline
%  \textbf{Number} & \textbf{Name} \\ \hline
%  1 & Nose \\ \hline
%  2 & Generator access tube  \\ \hline
%  3 & Adduction chamber  \\ \hline
%  4 & Kaplan head  \\ \hline
%  5 & Synchronous generator  \\ \hline
%  6 and 8 & Supporting structure \\ \hline
%  6 & Turbine access tube \\ \hline
%  7 and 9 & Combined bearing and Guide \\ \hline
%  10 & Distributor \\ \hline
%  11 & Rotor blade \\ \hline
%  12 & Cone \\ \hline
%  13 & Cube \\ \hline
%  14 & Suction tube \\ \hline
%  15 & Arc chamber \\
%  \hline
%\end{tabular}
%\captionof{table}{Bulb type turbine principle components}
%\caption{Componentes principais de uma turbina tipo bulbo}
%\label{tab::bulb_turbine}
%\end{center}

Turbine repair or inspection requires water flow stoppage and water drainage. In
Jirau, there is a 800 mm diameter access (bottom hatch) for rotor maintenance.
In the context of EMMA, the turbine points of interest are: 1) Propeller and
blades; 2) Arc chamber and adjacent areas; 3) Hatches; 4) Suction tube; and 5)
Available infrastructure.

%Atualmente, caso seja necessário algum reparo ou inspeção na turbina, é
% necessário que se interrompa o fluxo de água e que toda a água em seu interior seja drenada. Para manutenção do rotor, existe uma escotilha de acesso de diâmetro limitado. Entretanto, caso deseje-se realizar 
%a metalização de pás já instaladas, utilizando-se os processos atuais, é
%necessária a retirada de todo o aro câmara, desmontagem completa do rotor e
% logística de transporte das pás até o local onde a metalização será realizada. Essa operação, caso necessite ser realizada, demandaria a mobilização
%de diversas equipes de manutenção, operação de pórtico rolante e transporte,
%além de impossibilitar a utilização da turbina durante várias semanas.
%No contexto da solução proposta, os pontos de interesse da turbina são:

1) Propeller and blades: the rotor or turbine propeller consists of the hub, the
blades and the cone. Blades of Jirau's turbine is approximately 2.5 m tall
and 3 m wide, they are fully reachable from the turbine interior, excepting
the lips, which can be visualized through a small top hatch. The blades' angles
relative to the water flow can be from $0^o$ to $29^o$, with no overlapping
areas.
%figure \ref{fig::blades_angle}. These angles can be exploited to optimize
% working space for blade coating process and also influences the working region between the distributor and the rotor. 
Rotor position can be changed manually and be rotated in both directions without
limit. But blades' angles changes requires hidraulic actuators.
%However, this operation is an imprecise and risky task. Hence, the proposed
%solution should optimize rotations required for the blade processing. 
%The
% figure\ref{fig::blade_rijeza} exemplifies a turbine blade recently coated by Rijeza company. 
%O rotor ou hélice da turbina é constituído do cubo, as pás e o cone. 
%Nas turbinas da usina de Jirau, cada pá mede, aproximadamente, 2,5m de altura e
%3m de largura. A partir do interior da turbina, todas as superfícies da pá são
%alcançáveis, com exceção da borda e do lip da pá. O único ponto de acesso à
%essa regiâo é por meio da escotilha superior de acesso. A figura
%\ref{fig::blade_rijeza} exemplifica uma pá do rotor presente na usina de Jirau
% recém metalizada no galpão da Rijeza.

%\begin{figure}[h!]	
%	\includegraphics[width=\columnwidth]{figs/viagem/2015_04_28/Rijeza/img_4887}
%	\caption{Jirau turbine blade.}
%	\label{fig::blade_rijeza}
%\end{figure}



%A angulação de cada pá em relação ao fluxo d'água pode ser alterado em 29$^o$,
%14.5$^o$ para cada lado a partir da posição inicial, não havendo sobreposição
%entre as pás, como ilustrado na figura \ref{fig::blades_angle}.
%Essa angulação pode ser explorada para otimizar o espaço de trabalho necessário
%para o processamento da pá e também influencia o acesso à região
%entre o distribuidor e o rotor, uma vez que não existe acesso pela montante da
%turbina. Entretanto, vale observar que esta angulação não pode ser alterada
%manualmente e só pode ser realizada uma vez, antes do desligamento da turbina.
% A posição do rotor também pode ser manualmente alterada, possibilitando que o
%mesmo seja girado em ambas as direções e sem limite de revoluções. Entretanto,
%essa operação é uma tarefa imprecisa e envolve um certo risco às pessoas que a
%realizam. Sendo assim, a solução proposta deve otimizar o número de rotações
%necessárias para o processamento de todas as pás.

%\begin{figure}[h!]	
%	\includegraphics[width=\columnwidth]{figs/intro/blades_angle}
%	\caption{Example of the blade angle limits.}
%	\label{fig::blades_angle}
%\end{figure}

%\subsubsection{Ring chamber and adjacent regions}
2) Arc chamber and adjacent areas: the arc chamber and the distributor area
have metal surface, but only the latter two are composed of ferromagnetic materials, allowing magnetic fixing solutions for robotic
systems. However, the cylindrical and sloping shape of the arc camara hinders
robot fixation and movement. An horizontal plane or an efficient and robust base should be build for the system fixation.
Under turbine maitenance, devices and equipments are fixed by scaffolding
anchored by ropes. The figure~\ref{fig::andaime} illustrates a turbine under
maintenance.
%O aro câmara, assim como o a região próxima ao distribuidor e também ao tubo de
%sucção possuem superfícies metálicas. Essa característica possibilita a
%exploração de soluções de fixação magnética.

%Somente a região compreendida pelo aro câmara é plana e tendo como agravante a
% presença do distribuidor na região à montante ao rotor. É necessário que a inclinação presente nessas superfícies seja contabilizada e uma solução eficiente 
%de apoio ou plano elevado seja desenvolvida caso haja necessidade de fixação de
% alguma parte do sistema. Atualmente todo o trabalho é realizado por meio da montagem de andaimes ancorados por cordas. A
%figura \ref{fig::andaime} ilustra uma estrutura utilizada no modo de inspeção e
%manutenção atuais.

\begin{figure}[h!]	
	\includegraphics[width=\columnwidth]{figs/viagem/2015_04_28/UG/img_4969}
	\caption{Turbine under maintenance. Scaffolding as fixation points for
	equipments.}
	\label{fig::andaime}
\end{figure}

 
3) Hatches: the two turbine accesses are the 800 mm diameter bottom hatch,
located at the beginning of the suction tube and generally used by operators
for maintenance; and the 357 mm diameter top hatch, located at the top of the
arc chamber and generally used for blade lip inspection.
%O acesso à turbina se dá por duas escotilhas, uma inferior, localizada no
% ínicio do tubo de sucção próxima ao aro câmara e outra superior, localizada na parte superior do aro câmara.

%Operators access the turbine via the bottom hatch and all equipments for
%maintenance are transported through this hatch. The bottom hatch diameter is
%80 cm.
%A escotilha inferior é o acesso utilizado para a entrada de pessoas na turbina
% e todo material utilizado para reparos é transportado através dessa escotilha. Na usina de Jirau existem dois 
%tipos de escotilha de acesso inferior, sendo a menor delas possuindo 80cm de
% diâmetro.

%The top hatch is used mainly for blade lip visual condition inspection. The
%diameter of the top access is approximately $35.7cm$, thus no equipments
%are transported through this hatch. The figures~\ref{fig::esc_sup_ext} e
%~\ref{fig::esc_sup_int} illustrate the top hatch from different views.
%A escotilha superior é utilizada, principalmente, para a inspeção visual do
%estado dos Lips das pás.
%O diâmetro do acesso superior é de aproximadamente $35.7cm$, limitando as
%dimensões dos equipamentos que podem ser transportados através da escotilha. As
% figuras \ref{fig::esc_sup_ext} e \ref{fig::esc_sup_int} ilustram o acesso à escotilha superior pelo exterior ao
%aro câmara e a visão pelo interior da turbina,
%respectivamente.

%\begin{figure}[h!]	
%	\includegraphics[width=\columnwidth]{figs/viagem/2015_04_28/UG/img_4979_mod}
%	\caption{Top hatch view - ring chamber exterior}
%	\label{fig::esc_sup_ext}
%\end{figure}

%\begin{figure}[h!]	
%	\includegraphics[width=\columnwidth]{figs/viagem/2015_04_28/UG/img_4982}
%	\caption{Top hatch view - ring chamber interior}
%	\label{fig::esc_sup_int}
%\end{figure}

4) Suction tube: at the end of the discharge pipe is located the downstream
stoplogs and then the riverbed. If the stoplog are not inserted, there is a 10 m wide gap, which
could be used as access. However, the high water flow due to the
openning of the distributor make it impossible for access. The distributor is
not closed immediately due to environmental issues, since this becomes the
passage of fish.
%Ao final do tubo de descarga está localizado o vão dos stoplogs 
%de jusante ou da comporta vagão e, em seguida, o leito do rio. Caso os stoplogs 
%não estejam inseridos, existe um vão de, pelo menos, 10 m de largura. Porém,
% não é válida a utilização deste vão como acesso à turbina, pois há grande fluxo de
%água devido à abertura do distribuidor. O distribuidor não é fechado
%imediatamente por questões ambientais, já que este é o escoamento de peixes.

%criando assim
%um acesso extra para um sistema submarino. A figura \ref{fig::tubo_suc}
%exemplifica a magnitude do tamanho do acesso, deixando claro que o limitante de
%tamanho do sistema para a utilização desse acesso é o vão de entrada do
% stoplog, ilustrado na figura \ref{fig::stoplog}. Outra alternativa é utilizar um
%guindaste e submergir o sistema pelo próprio rio, entretanto o sistema ficaria
%sujeito as condições do ambiente.

%\begin{figure}[H]	
%	\includegraphics[width=\columnwidth]{figs/viagem/2015_04_30/Vao/img_5086}
%	\caption{Suction tube openning under construction.}
%	\label{fig::tubo_suc}
%\end{figure}

5) Available infrastructure: after the turbine drainage, maintenance provides
electricity and compressed air, requirements for the HVOF process. Other important factor is the presence
of a gantry crane outside the turbine, which could position the HVOF equipment
nearby the top hatch. %Also, it is possible to access the top hatch through the
% gantry crane.
%É importante ressaltar a infraestrutura dísponível para o desenvolvimento da
% solução.
%Após secar a turbina, é possível a disponibilização de energia elétrica e ar
%comprimo em seu interior, ambos importantes para o processo de metalização.
% Outro fator importante é a presença de um pórtico rolante que tem acesso até o andar diretamente 
%inferior ao aro câmara, posicionando todo o equipamento necessário nas
% proximidades da escotilha de acesso inferior. É possível também o acesso direto, por meio de pórtico, 
%à escotilha superior.

%\begin{figure}[h!]	
%	\includegraphics[width=\columnwidth]{figs/viagem/2015_04_28/UG/img_4989}
%	\caption{Gantry crane and top hatch}
%	\label{fig::portico}
%\end{figure}

%The environment may be briefly characterized by the blades dimensions,
%element to coated; the ring chamber structure, which limits
%the workspace of the robot; and the access.
%O ambiente pode ser resumidamente caracterizado pelas dimensões das pás,
%elemento a ser processado; características do aro câmara, estrutura que limita
% o espaço de trabalho do robô; e pelos acessos nos quais o sistema terá que
%utilizar.

%\begin{itemize}
%  \item \textbf{Turbine blades} - 420 stainless steel. Dimensions 2.5 x 2.5 m;
%  \item \textbf{Ring chamber} - cylindrical structure 3.95 radius and metal
%  surface;
%  \item \textbf{Access}: 
%  	\begin{itemize}
%    	\item Top hatch - 35 cm diameter;
%  		\item Bottom hatch - 80 cm diameter;
%  		\item Suction tube - 20 x 20 m, access by river. 
%  	\end{itemize}
%\end{itemize}

A 3D CAD model of the turbine was built with
\textregistered{SolidWorks} for future simulation and conceptual
solution analysis (figura~\ref{fig::ambiente3d}). The model is not fully detailed, but the upstream tunnel, the stator, the rotor, a small
sector of the downstream, and the hatches are represented with great accuracy.
%Os estudos das possíveis soluções exigiu uma visualização mais detalhada do
%volume livre no interior da turbina. Para isso, foi recriado o ambiente da
%turbina em CAD 3D no SolidWorks, a partir dos desenhos 2D de seção da turbina
%fornecidos pelo cliente.
%O modelo tridimensional do aro câmara permite o estudo e o dimensionamento
%% geométrico de alcance do manipulador para cada solução. Não foram necessários
%detalhamentos de todos os componentes, podendo ser apenas considerados, e
%representados com maior precisão, os perfis externos do túnel à montante, o
%estator, o rotor e uma pequena região à jusante, além dos acessos
%por escotilha superior e inferior.
%A figura~\ref{fig::ambiente3d} apresenta o ambiente da turbina em CAD e os
%possíveis acessos para realização das intervenções.

\begin{figure}[h!]
\centering
	\includegraphics[width=\columnwidth]{figs/estudo/solid/ambiente_3d} 
	\caption{3D CAD model of the Jirau turbine (\textregistered{SolidWorks}).}
	\label{fig::ambiente3d}
\end{figure}





\subsection{Robot tasks}\label{desc_taref}
This subsection describes the basic tasks of the robot for \textit{in situ}
turbine coating. %Generally, the robot should be able to perform the coating
% task as if the blade were not installed in the tubina, as it is done by Rijeza
%company, in an autonomous way. 
As mentioned, the turbine blade should conform
to the template, hydraulic profile, before the coating process. Therefore,
another task is to perform the hydraulic profile mapping, build a 3D model, and
analyze flaws.
%Esta subseção descreve as tarefas básicas do robô para o revestimento de
%turbinas \textit{in situ}. Em linhas gerais, o robô a ser desenvolvido deve ser
%capaz de realizar a tarefa de revestimento tal qual seria feita caso a pá não
% estivesse instalada na tubina e de uma maneira autônoma. A pá, antes de ser submetida ao
%processo de revestimento, deve estar em conformidade com o gabarito, perfil
% hidráulico de uma pá intacta. Portanto, uma tarefa do robô é realizar o mapeamento do perfil
%hidráulico, construir um modelo 3D e analisar imperfeições.

In case of deep blade deformations,% caused by cavitation and abrasion, 
the repair by welding should be done.% manually or automatically. 
An operator can manually perform the welding as there is no hard restrictions as the coating
process (accuracy, speed, load). However, the hostile and confined environment
can hinder the manual execution, thus welding can be also a robot task.
%Em caso de deformações, causados por cavitação e abrasão, estas precisam
%ser removidas manualmente ou de forma automatizada, possivelmente por
%soldagem. A tarefa de soldagem pode
%ser realizada por operador, manualmente, por não possuir todas as restrições
%da tarefa de revestimento (velocidade, precisão, carga e etc), porém o ambiente
%pode dificultar a operação de forma que a execução por um robô seja
%indispensável. 

If the blades conform to the hydraulic profile, the coating erosion
identification is made measuring the thickness of specific points on the
surface of the blade. An operator with a specific device can manually do this
process, efficiently in ten minutes. The coating erosion identification is not a
robot task.
%Após as pás estarem de acordo com o gabarito, faz-se a
%identificação do desgaste do revestimento, medindo sua espessura em pontos
%pontos específicos sobre a superfície da pá. Manualmente esse
%processo é realizado eficientemente em 10 min, justificando a não necessidade
% de esta ser uma tarefa do robô. 

The sandblasting process is normally required before the coating
operation.%, subsection~\ref{sandblasting}.
Rijeza company performs the sandblasting manually, but there are studies and
companies performing sandblasting with robots. Thus sandblasting could be a robot task.
%Em caso de necessidade de aplicação
%de novo revestimento, é necessária a remoção do revestimento antigo por
%jateamento, a fim de deixar a superfície rugosa e aumentar sua aderência. A
%tarefa de jateamento é atualmente realizada maualmente, mas também pode ser
%realizada pelo robô. Como ambos os lados da pá são revestidos, o jateamento
% deve ser realizado em ambos os lados. Vale ressaltar que, em teoria, pode-se aplicar revestimento por metalização sem retirar o último revestimento,
%porém esse processo ainda se encontra em fase de estudos na Rijeza.
%Segue-se o exemplo de empresas de aviação, onde existe a
%prática de retirar todo o revestimento antigo antes de aplicar o novo.

%Por fim, o robô deverá aplicar o revestimento como 
%forma de prevenir o dano causado pelos fenômenos abrasivos. O robô projetado
%para fazer o revestimento precisa preencher todos os requisitos discutidos na
%subseção~\ref{sec::desc_hvof} e ser adaptável ao ambiente, cujos as restrições
%são discutidos na subseção~\ref{sec::desc_contex}. 

Below, a summary of the task to be performed:
Tasks that can be performed manually: 1) Blade damage inspection, for repair and
for coating; 2) Repair; 3) System mounting ; 4) Surface sandblasting.
%Das tarefas a serem relizadas, são destacadas as seguintes:
%Tarefas que podem ser executadas manualmente:
%\begin{itemize}
 % \item Blade damage inspection, for repair and for coating.%Inspeção e
  % análise de danos na pá, tanto para reparo quanto para revestimento.
 % \item Repair.
 % \item System mounting.%Montagem do sistema.
 % \item Surface sandblasting.%Jateamento da superfície.
%\end{itemize}

Tasks that can be performed by robot: 1) Hydraulic profile modeling; 2)
Calibration; 3) Sandblasting; 4) Repair; 5) HVOF coating.
%Tarefas que poderão ser executadas
% pelo robô:
%\begin{itemize}
%  \item Hydraulic profile modeling.%Modelar o perfil hidráulico.
%  \item Calibration.
%  \item Sandblasting.
%  \item Repair.
%  \item HVOF coating.
%\end{itemize}




\section{State of the art}\label{sota}
 
The study of the state of the art of robots for HVOF coating process on
hydraulic turbine blades covers systems that meet some of following requirements:
operation in hostile and confined environments; object manipulation with
at least 8.5 kg payload, 5 mm precision and $0.67 m/s$ speed; 2.5 m x 3.0 m
system workspace; and the ability to operate on complex 3D geometry surfaces.
The robots were divided by fixation technologies.
%O estudo do estado da arte de robôs para a realização de HVOF em pás de
% turbinas hidráulicas contempla os sistemas que atendem a alguns dos requisitos: operar
%em ambientes de alta periculosidade; capacidade de carga para os dispositivos
% HVOF; manipular a pistola HVOF com velocidade de $0.67 m/s$; precisão de 5mm; ter
%área de trabalho de 2.5 m x 2.5 m; e operar sob superfícies 3D de geometria
%complexa. As soluções foram divididas em subseções de acordo com as tecnologias
%de fixação dos robôs.


 
\subsection{Robots on rails}\label{sec::rail}
In industry, HVOF coating processes are performed by robotic manipulators,
which offer tasks versatility and large workspace, required for this
type of application. Meanwhile, a robotic arm capable of coating all the turbine
blade surface on a fixed position is not compact or mobile, and hard to be
mounted/unmounted. A prismatic joint coupled to a rail is a common strategy for
extending the robot working space, without adding weight to the manipulator,
and the rail can use the structures in the environment as a support.
%Na indústria, a automatização de processos de metalização, é
%normalmente realizada com a utilização de manipuladores robóticos, pois oferece
%a versatilidade de tarefas e espaço de trabalho necessários para esse
%tipo de aplicação. Entretanto, um sistema composto por um braço robótico capaz
%de operar em toda a extensão da superfície da pá da turbina hidrelétrica
%não é compacto, nem móvel o suficiente para ser instalado e desinstalado para a
%operação de manuntenção \textit{in-situ}.


%A introdução de uma junta prismática acoplada a um trilho é uma estratégia para
%reduzir o tamanho e o peso de um manipulador robótico.  Assim, é possível
% estender o espaço de trabalho do robô, sem adicionar peso ao manipulador, uma vez que o
% trilho pode usar as estruturas presentes no ambiente como apoio. 


%Na literatura foram encontradas duas soluções para aplicações de manutenção e
%inspeção, como solda, específicas para o contexto de turbinas hidráulicas. As
%aplicações diferem, principalmente, na estratégia de fixação do sistema
%de trilhos.
%O Roboturb \citep{roboturb} realiza a fixação
%diretamente na pá do rotor, enquanto o robô Scompi \citep{scompi} utiliza um
% trilho fixado em
%estruturas adjacentes à pá ou peça a ser reparada.

The Roboturb \citep{roboturb} is a robotic manipulator composed of six
revolution joints and one prismatic joint coupled to a flexible rail.
%figure~\ref{fig::roboturb}. 
The robot performs welding, filling cavities generated by erosion. The rail
may be shaped and then fixed to the blade surface by a passive system of
suction cups. The robot has two end-effectors: an optical sensor for erosion
inspection; and a welding tool, a PWH-4A plasma torch with automatic feeder.
%O Roborturb \citep{roboturb} consiste em um manipulador robótico com seis
% juntas de revolução e uma junta primsática acoplada a um trilho flexível, como pode ser observado
%na figura \ref{fig::roboturb}, utilizado para o preenchimento de cavidades
%geradas por cavitação.
%O trilho pode ser conformado e, então, fixado à superfície da pá por meio de um
%sistema passivo de ventosas ou ímãs. O robô tem a possibilidade de utilizar
% dois efetuadores distintos, o primeiro consiste em um sensor ótico para inspeção do 
%estado de erosão da pá e o segundo consiste em uma ferramenta de solda do 
%tipo tocha plasma PWH-4A com alimentador automático de arame, responsável pelo 
%depósito de solda para o preenchimento das cavidades identificadas pelo
% sistema.


%TODO Abelha: Posicionar corretamente as figuras
%    \begin{figure}[h!]	
%		\includegraphics[width=\columnwidth]{figs/trilhos/roboturbpaper}
%		\caption{Roboturb - Robotic manipulator on rail}
%		\label{fig::roboturb}
%	\end{figure}
%	\begin{figure}[h!]
%		\includegraphics[width=\columnwidth]{figs/trilhos/scompi}
%		\caption{SCOMPI - Manipulador robótico sobre trilhos rígidos}
%		\label{fig::scompi}
%	\end{figure}

The Scompi robot \citep{scompi}%, figure~\ref{fig::scompi}, 
is a multipurpose manipulator, designed to perform repairs on \textit{Francis}
type turbines, as welding and grinding. It has six degrees of freedom: a robotic
manipulator with five revolution joints; and a prismatic joint, coupled to
curved rails that are designed specifically for each application.
%Por sua vez, o robô Scompi \citep{scompi}, fig \ref{fig::scompi}, é um
%manipulador multipropósito projetado para realizar reparos em turbinas do tipo
% \textit{Francis}, como solda e esmerilhamento das pás. O sistema possui seis graus de liberdade,
%sendo consitituído por um braço robótico com cinco juntas de revolução, e o
%último grau de liberdade proveniente de uma junta prismática que percorre um
% sistema de trilhos retos ou curvos que são projetados para cada aplicação especificamente. 


%Sistemas baseados em trilhos tem como maior benefício a redução do tamanho e,
%consequentemente, do peso do manipulador necessário para a execução de tarefas
%em um espaço de trabalho que englobe toda a superfície da pá.
%Essa redução proporciona facilidade de transporte do robô até o interior da
%turbina e também possibilita o projeto de manipuladores que tenham a rigidez
%necessária para a realização das tarefas desejadas. 

Fixed robotic manipulators which meets the HVOF payload and
workspace requirements would be too heavy. Systems based on rails fixed
on the blade itself, require rail handling as the entire
blade surface should be coated and the area in which the rail is fixed is not in
the robot workspace. In addition, rail systems fixed on adjacent structures
should considerate the installation conditions and balance the cost-benefit
of installation/removal rail, and robustness.
%Manipuladores robótico fixos, rígidos o suficiente para aguentar as forças
% intrínsecas ao processo de metalização e com espaço de trabalho necessário para trabalhar em
%toda a extensão da superfćie da pá seriam muito pesados.
%Entretanto, sistemas baseados em trilhos com fixação na própria pá do rotor,
% necessitam que o trilho seja movido caso se deseje que toda a superfície da pá sofra
%manuntenção, uma vez que a área em que o trilho está apoiado não pertence ao
% espaço de trabalho do robô. Em adição, sistemas com fixação de trilhos nas estruturas
%adjacentes à pá devem atentar as condições para a instalação disposta pelo
%ambiente para equilibrar a relação de custo benefício entre facilidade de
%instalação/remoção do trilho e a robustez.
The system advantages are: 1) manipulator size reduction; 2) manipulator
weight reduction. The disadvantages are: 1) rail mounting and unmouting; 2) rail
handling if fixed on the blade. 
\subsection{Climbing robots}\label{sota_climbers}
Climbing robots are systems capable of supporting its own weight against
gravity, moving in simple or complex geometric structures, such as
walls, ceilings and roofs, turbine blades and nuclear plants.
Climbers provide operational efficiency in hazardous environments, and increase
operators health and safety. Some applications for climbing robots are:
skyscrapers inspection and cleaning, storage tanks diagnosis in nuclear
power plants, shell ships and turbines welding and
maintenance \citep{armada2003application}.
%Robôs escaladores são sistemas capazes de sustentar seu próprio peso contra a
%gravidade, movendo-se em simples ou complexas estruturas geométricas, como
%paredes, tetos e telhados, palhetas de turbinas e plantas nucleares.
%Essa classe de robôs oferece eficiência operacional em ambientes
%de alta periculosidade, sendo utilizados visando saúde e segurança dos
%trabalhadores, como em inspeção e limpeza de arranha-céus, diagnóstico de
%tanques de armazenamento em plantas nucleares, solda e manutenção de cascos de
%navios e palhetas de turbinas \citep{armada2003application}. 

The development major challenges for climbers are mobility, adhesion, power
consumption, load capacity, and weight. In \cite{modular} and \cite{climbsurv},
climbers are divided into types of locomotive mechanisms: legs; walker; translation; wheels; tracks;
advance by arms; cable-driven; and biomimetics. And adhesion types:
suction or pneumatic; magnetic; electrostatic; chemical; gripping; and hybrid.
%Os grandes desafios nos projetos de sistemas escaladores são mobilidade e
%aderência, além de consumo de energia, capacidade de carga e peso. Em
%\cite{modular} e \cite{climbsurv}, os robôs escaladores são divididos em tipos
%de locomoção:
%pernas; como andador; utilizando segmentos deslizantes; rodas; esteiras; avanço
%pendurado por braços; por cabos; e biomimética. E categorias de adesão: sucção
%ou pneumática; magnética; eletrostática; química; preensão; e híbrida.

In the specific case of the HVOF coating problem in hydropower turbines, the
following climber robots should be investigated, as all of them have millimetric
accuracy, and robustness:
%No caso específico deste estudo da arte, destacam-se os robôs escaladores com
% as seguintes aplicações:

\begin{itemize}
  \item \emph {ships and turbines}: RRX3 for welding
   \citep{rrx3}, \emph{Climbing Robot for Grit Blasting for cleaning}
   \citep{crgb} and ICM Robot for inspection \citep{icm};
  \item \emph{Industrial}: ROME II \citep{roma} and CROMSCI \citep{CROMSCI}, both for inspection;
  \item \emph{petrochemical plant}: TRIPILLAR \citep{tripillar} for inspection.
\end{itemize}

The RRX3, Daewoo Shipbuilding \& Marine Engineering, is a
robot for hull ship welding with manipulator. Its adhesion is gripping
type, locomotion is translation type (sliding segments), and longitudinal locomotion by wheels.
The RRX3 robot has a 1.5 m manipulator with three prismatic joints and three
revolution joints  (3P3R) for welding operation. The system weighs 120 kg with 5
kg payload, it has a manipulator with welding tool, but low speed end effector.
Regarding the HVOF application, the locomotion type is not efficient.
%O RRX3 (figura~\ref{rrx3}), Daewoo Shipbuilding and Marine Engineering, é um
%robô para a soldagem de casco de navios. Possui adesão por preensão, locomoção
% transversal utilizando segmentos deslizantes e locomoção longitudinal por rodas. Possui um manipulador de 1.5 m com três juntas
%prismáticas e três juntas de revolução (3P3R) para a operação de soldagem. 

%RRX 3 main characteristics are: base and manipulator with
%120 kg and 5 kg of payload, respectively; accurately
%millimeter manipulator and low speed end effector; robustness; welding tool
%operation; limited translation type locomotion.
%As características principais do robô são: base e manipulador com
%capacidades de carga de 120 kg e 5 kg, respectivamente; manipulador com
% precisão milimétrica e efetuador de baixa velocidade; robustez para operar em ambiente de
%alta periculosidade; opera instrumento de solda; e locomoção transversal é
%restrita à aplicação.

%\begin{figure}[ht]
%\centering
%\includegraphics[width=8.4cm]{figs/climbers/RRX3_moving.jpg}
%\caption{RRX3 translation locomotion.}
%\label{rrx3}
%\end{figure}

The \emph{Climbing Robot for Grit Blasting}, %(figura~\ref{grit}), 
University of Coruna, is a robot for ship abrasive blasting. The robot moves by
two sliding platforms with magnetic adhesion. The platforms have relative motion
between them and can rotate to compensate ship's hull curvatures or to
deflect objects. The abrasive system is similar to HVOF, but the robot locomotion not applicable to complex structures.
%O \emph{Climbing robot for Grit Blasting} (figura~\ref{grit}), University of
%Coruna, é um robô para jateamento abrasivo em navios. O robô utiliza duas
% plataformas deslizantes com sistema de adesão por ímã magnético. Os módulos apresentam movimentação relativa entre si e pode rotar
%para compensar as curvaturas do casco do navio ou desviar de objetos. 

%The \emph{Climbing robot for Grit Blasting} main characteristics are:
%abrasive system similar with HVOF; accurately
%millimeter movement; locomotion not applicable to complex structures; no
%manipulator.
%As características principais do robô são: base com
%capacidade de carga de sistema abrasivo semelhante a HVOF; base com
%locomoção de precisão milimétrica; locomoção ampla, mas não aplicável a
%estruturas complexas; e não possui manipulador, sendo necessário percorrer todo
%o casco.

%\begin{figure}[ht]
%\centering
%\includegraphics[width=8.4cm]{figs/climbers/grit.png}
%\caption{Climbing robot for Grit Blasting}
%\label{grit}
%\end{figure}


\emph{The Climber}%(figure~\ref{icm})
, ICM Robotics, is an inspection robot for
wind turbines, coating removal, surface cleaning and coating application.
It has pneumatic adhesion and locomotion by tracks. It has 25 kg base payload,
and a small sized low speed manipulator. The locomotion type presents
restrictions to some curvatures.
%\emph{The Climber} (figura~\ref{icm}), ICM Robotics, é um robô para inspeção de
%turbinas eólicas, remoção de revestimento, limpeza de superfície, e aplicação
% de revestimento.
%Possui adesão pneumática (sucção) e locomoção por esteiras. 

%\emph{The Climber} main characteristics are: 25 kg base payload; accurately
%millimeter movement; a modular manipulator can be attached to the base; small
%size manipulator and low speed; locomotion type presents restriction to
%some curvatures.
%As características principais do robô são: base com capacidade de carga de 25
%kg; base com locomoção de precisão milimétrica; manipulador modular pode ser
%acoplado à base; manipulador de dimensão reduzida e baixa velocidade; e
%locomoção apresenta restrição a algumas curvaturas acentuadas.

%\begin{figure}[ht]
%\centering
%\includegraphics[width=8.4cm]{figs/climbers/icm.png}
%\caption{The Climber}
%\label{icm}
%\end{figure}

The Rome II%(figura~\ref{roma2})
, University Charles II of Madrid, is an inspection robot for complex
environments. It has pneumatic adhesion and moves like a caterpillar
(biomimicry). Rotation and planning trajectory are performed optimally to
ensure stability and obstacle avoidance.
%O ROMA II (figura~\ref{roma2}), Universidade Carlos II de Madrid, é um robô
% para inspeção de ambientes complexos. A sua tecnologia de adesão é pneumática (sucção) e
%locomove-se como uma lagarta (biomimética). Sua movimentação e planejamento de
%trajetória são realizados de maneira ótima de forma a garantir estabilidade e
%evitar obstáculos. 

%The Rome II main characteristics are: high payload capacity; accurately
%millimeter movement; no manipulator; locomotion for complex environments.
%As características principais do robô são: base com grande capacidade de carga;
%base com locomoção de precisão milimétrica; não possui manipulador; locomoção
% em ambientes de grande complexidade.

%\begin{figure}[ht]
%\centering
%\includegraphics[width=8.4cm]{figs/climbers/roma2.jpg}
%\includegraphics[width=4.2cm,height=4.2cm]{figs/climbers/roma2.jpg}
%\caption{ROMA II.}
%\label{roma2}
%\end{figure}
CROMSCI% (figure~\ref{cromsci})
, Kaiserslautern University of Technology, is an
inspection and autonomous robot for large concrete walls, as
pillars of bridges and dams. Its adhesion system is composed of seven vacuum
chambers (suction), valves and pressure sensors for system control. The
locomotion system has omnidirectional wheels.
%CROMSCI (figura~\ref{cromsci}), Kaiserslautern University of Technology, é um
%robô autônomo para inspeção de grandes paredes de concreto, como pilares de
% pontes, barragens. Seu sistema de adesão é composto por sete câmaras de vácuo (sucção), com um sistema
%de controle por válvulas e sensores de pressão para reagir rapidamaente a
%condições adversas. Locomove-se com rodas omnidirecionais para locomoção.

%The CROMSCI main characteristics are: low payload capacity; accurately
%millimeter movement; no manipulator; low speed.
%As características principais do robô são:
%base com pouca capacidade de carga; base com locomoção de precisão milimétrica;
% não possui manipulador; e apresenta baixa velocidade.

%\begin{figure}[ht]
%\centering
%\includegraphics[width=8.4cm]{figs/climbers/cromsci.jpg}
%\caption{The CROMSCI robot.}
%\label{cromsci}
%\end{figure}

TRIPILLAR, Ecole Polytechnique Federale de Lausanne, is a small inpection robot
(96 x 46 x 64 mm) for petrochemical plants. Its adhesion is done by magnetic
legs on a caterpillar triangular shape, and it moves by tracks.

A specific case of climbers is the \textit{cabling robots}, which use a set
of cables to ensure its proper positioning in its working area. The cables provide
manipulator range improvement, decrease the adhesion complexity and reduce the
weight carried by the robot. As an example, the \textit{torboMate} is a climber
with magnetic adhesion, it can have two or more emitting jets with 4000 bar of
supply capacity. It has 45 kg and reaches 20 m/min speed \citep{torbo}.

RIWEA is a purely cabling robot, as it has no other type of position
adjustment, for wind power turbines cleaning. It is an open frame concept robot
which uses four ropes to move up and down. It has five main parts, which
automatically adjust to the blade surface during its move
\citep{jeon2012maintenance}. Its greatest strength lies the ability to adapt the
curvature of the blade while maintaining a foothold on it, and it is also less
susceptible to vibration \citep{riwea}.

Climbing robots are widely applicable, have different adhesion solutions and
mobility. There is not, so far, a climber that fulfills all the HVOF
requirements for the turbine blade coating, but some of the systems, such as
\emph{The Climber} (ICM Robotic), can generate complete solutions with
adaptations.
%Os robôs escaladores são utilizados em diversas aplicações e possuem diferentes
%soluções de aderência e locomoção, como foi exposto nesta subseção. Não há,
%até o momento, um robô escalador que possui todas as características
%exigidas para a tarefa de HVOF em pás de turbinas, porém a adaptação de
%alguns desses sistemas, como \emph{The Climber} da ICM Robotic, pode gerar
%soluções completas.

The advantages for climbing robot solution are: easily installation, small sized
manipulator, small base, lightweight, autonomy; and the disadvantages are:
complex locomotion system, complex mechanics, manual installation on blade,
well-developed robot safety system, limited battery or umbilical management
system. 
%As vantagens e desvantagens para solução de robôs escaladores são:
%\textbf{Advantages:}
%\begin{itemize}
%  \item Easily installation;
%  \item Small size manipulator, since robot moves on blade;
%  \item Small base;
%  \item Small weight;
%  \item Autonomy while operating; 
%\end{itemize}
%and disadvantages 
%\textbf{Disadvantages:}
%\begin{itemize}
%  \item Complex locomotion system with obstacle avoidance and path
%  planning;
%  \item Complex mechanics, as robot should be able to support its weight plus
%  the manipulator and the HVOF spary gun;
%  \item Robot must be manually installed on each blade or a complex locomotion
%  system by arms should be developed;;
%  \item Robot safety system must be well developed;
%  \item Limited battery or umbilical management system for mobile robots;
%\end{itemize} 
\subsection{Cabling robots}
Cabling robots are systems which use one set of cables to or ensure its proper
positioning in its working area. Cabling robots may have other
adhesion methods technologies in combination with cables.
%São classificados como robôs cabeados quaisquer sistemas robôticos que façam
%uso de um conjunto de cabos e/ou cordas para auxiliar ou mesmo garantir seu
%posicionamento adequado na sua região de trabalho. Sendo assim, robôs cabeados
%podem possuir outros métodos de fixação em conjunto com seu cabeamento.

The cabling system is ideal to applications which robot locomotion is
restricted to a vertical plane and speed velocities are not required. The
cabling system is an weight reduction and manipulator range improvement, or
decrease the adhesion complexity of a climbing robot.
%A idéia do uso de um sistema de cabos surge naturalmente quando o deslocamento
%se mostra majoriamente restrito a um plano vertical e não há exigência de
%grandes velocidades de deslocamento. O sistema é usado como forma de reduzir o
%preso e melhorar o desempenho de um braço mecânico de mesmo alcance, ou
% diminuir a complexidade e a força de aderência necessária para um escalador.

As an example, the \textit{torboMate} is a climber with magnetic
adhesion, can have two or more emitting jets with 4000 bar of supply capacity.
It has 45 kg and reaches 20 m/min speed \citep{torbo}.
%Para exemplificar essa categoria foram selecionados dois robôs. O
%\textit{torboMate} é um escalador que possui adesão magnética que o
%permite caminhar livremente. Pode ter dois ou um
%emissor de jatos com capacidade para abastecimento em até 4000 bar. Possui 45
% kg e atinge uma velocidade de até 20 m/min \citep{torbo}.

\begin{figure}[ht]
	\centering
	\includegraphics[width=8.4cm]{figs/cables/torbo}
	\caption{TorboMate "Crawler" robot}
	\label{fig:cables:torbo}
\end{figure}

RIWEA is a purely cabling robot, as it has no other type of position
adjustment, for wind power turbines cleaning. It is an open frame concept robot
which uses four ropes to move up and down. It has five main parts automatically
adjust to the blade surface during its move \citep{jeon2012maintenance}.
However, there are chances where RIWEA robot can misjudge contaminated parts of
the blade as cracks. Its greatest strength lies the ability to adapt the
curvature of the blade while maintaining a foothold on it, and it is also less
susceptible to vibration \citep{riwea}.
%RIWEA é um robô puramente cabeado, no sentido em que ele não possui nenhum
%outro tipo de forma de ajuste de posição além do sistema de cabos. É um
%conceito de robô de estrutura aberta que faz uso de quatro cordas para se
%deslocar verticamente \citep{jeon2012maintenance}. Seu maior diferencial reside
%na capacidade de se adaptar a curvatura da pá mantendo sempre um ponto de apoio
%sobre ela, sendo também menos suceptivel a vibrações \citep{riwea}.

\begin{figure}[!h]
	\centering
	\includegraphics[width=8.4cm]{figs/cables/riwea}
	\caption{RIWEA robot}
	\label{fig:cables:riwea}
\end{figure}

The advantages and disadvantages for cabling robot solutions are:

\textbf{Advantages:}
\begin{itemize}
  \item Load reduction on robot / higher payload capacity.
  \item Robot reach can be extended at low cost.
\end{itemize}

\textbf{Disadvantages:}
\begin{itemize}
  \item Cabling management system complexity.
  \item Need of a superior fixation point for cabling.
\end{itemize}



\subsection{Manipulador com base esférica}
A research and development project was presented in
\cite{motta2010prototype} to propose methodology, simulation and
the steps for constructing a robotic system to recover damage
in hydraulic turbines blades. The robotic system does
repair using the arc welding technology before held
manually in highly dangerous environments with temperatures ranging
between $ 40^o C$ and $ 99^o C$, in a 10 hours operation.
%Um projeto de pesquisa e desenvolvimento foi apresentado em
%\cite{motta2010prototype} com o objetivo de propor metodologia, simulação e
%os passos para construção de um sistema robótico para recuperar danos materiais
%em pás de turbinas hidráulicas. O sistema robótico faz
%reparo utilizando a tecnologia de soldagem a arco elétrico, antes realizada
%manualmente em ambientes de alta periculosidade com temperaturas que variam
%entre $40^o C$ e $99^o C$, e operações que duram em torno de 10 horas. 

The robot must meet the following requirements:
\begin{itemize}
   \item ability to operate in any position: horizontal, vertical,
   reversed;
   \item light weight for portability and blade fixation;
   \item stiffness to deflection: payload on the wrist occurs in any
   direction and extension;
   \item high precision;
   \item parts availability on the market;
   \item user interface;
   \item large workspace;
   \item adhseion to the hydraulic turbine blades.
\end{itemize}

%O robô deve atender aos seguintes requisitos:
%\begin{itemize}
%  \item Capacidade de operar em qualquer posição: horizontal, vertical,
%  invertida;
%  \item Pouco peso para portabilidade e fixação às pás;
%  \item Rigidez à deflexão: carga no punho do manipulador ocorre em qualquer
%  direção e extensão;
%  \item Grande precisão;
%  \item Disponibilidade de peças no mercado;
%  \item Controle com interface de usuário;
%  \item Grande área de trabalho;
%  \item Facilidade de adesão às pás de turbinas hidráulicas.
%\end{itemize}

The solution to the robotic system has spherical topology as can be
shown in figure~\ref{fig: spherical} and the following characteristics:
%A solução para o sistema robótico apresenta topologia esférica, como pode ser
%visto na figura~\ref{fig:esferico} e características:

\begin{figure}[ht]
\centering
\includegraphics[width=8.4cm]{figs/esferico/esferico.jpg}
\caption{Illustration of the robotic system with spherical base}
\label{fig:esferico}
\end{figure}

\begin{itemize}
   \item manipulator with three DOF (2R1P) and wrist with two DOF (2R);
   \item 3D surface mapping with laser scanner;
   \item embedded electronics;
   \item arc welding;
   \item blade adhesion by magnetic or suction devices;
   \item low cost;
   \item ring-shaped workspace with 2.5 m, and 60 cm height;
   \item 30 kg weight and dimensions 30 x 25 x 100 cm;
   \item autonomous manipulator;
\end{itemize}

%\begin{itemize}
%  \item Três (3) graus de liberdade no manipulador (2R1P) e dois graus de
%  liberdade no punho (2R);
%  \item Mapeamento de superfície 3D com laser;
%  \item Eletrônica embarcada;
%  \item Soldagem por arco elétrico;
%  \item Fixação nas pás por dispositivos magnéticos ou de sucção;
%  \item Baixo custo;
%  \item Área de trabalho em forma de anel com 2.5 m e 60 cm de altura;
%  \item Peso 30 kg e dimensões 30 x 25 x 100 cm;
%  \item Robô com manipulador autônomo;
%\end{itemize}

The robotic manipulator system with spherical base is a compatible solution
for the HVOF application in hydraulic turbine blades, since its
original application is blade repair by arc welding, same environment and
similar challenge. All the system advantages and features are applicable to
the solution of a HVOF system. However, there are particular challenges in the
HVOF process, which are disadvantages of the solution:

%O sistema robótico de manipulador com base esférica apresenta solução
% compatível para a aplicação de HVOF em pás de turbinas hidráulicas, já que sua aplicação
%original é soldagem das pás, semelhante ao desafio deste artigo. Todas as suas
%características são vantagens e aplicam-se à solução de um sistema para HVOF.
%Há, porém, desafios particulares na metalização das pás e que são desvantagens
%da solução:


\textbf{Disadvantages:}
\begin{itemize}
   \item The HVOF must be performed in the entire blade. Therefore, the
   positions of the system must be manually changed at least 2 times per blade
   side, and must be moved blade to blade;
   \item The end effector must process the blade with great speed, as
   required by the HVOF.
\end{itemize}

%\textbf{Desvantagens:}
%\begin{itemize}
%  \item A metalização deve ser realizada em toda a pá. Portanto, o sistema
%  deverá ser manualmente trocado de posição, pelo menos 4 vezes (duas posições
%  para a frente e duas posições para a região de trás). E deve ser trocado de
  % pá em pá;
%  \item O efetuador deve percorrer a pá com grande velocidade, como exige o
%  processo de metalização.
%\end{itemize}

\section{Design of an autonomous robot for \textit{in situ} HVOF
process}\label{sec:projeto}

The autonomous system design for HVOF in hydraulic turbine blades include
solutions that meet all the application's requirements. Thus, the envisioned
robots of this section merge some technologies exhibited in
section~\ref{sec:sota} in the context of the Jirau hydroelectric dam.

%O projeto de robôs autônomos para HVOF em pás de turbinas hidráulicas contempla
%as soluções que atendem a \textbf{todos} os requisitos da aplicação. Dessa
%forma, serão idealizados robôs com a fusão das tecnologias expostas na
%seção~\ref{sota} e no contexto da usina hidrelétrica de Jirau. 

In section~\ref{sec::consideracoes}, the arc chamber accesses were described
and their restrictions are essential for the elaboration of the solution.
This section is divided into robotic solutions for both accesses, since they
are the most important development restriction, as they limit robot's
dimensions, features, and demand different logistics.

%Na seção~\ref{sec::consideracoes}, os acessos ao aro câmara foram
%descritos e suas restrições são fundamentais para a elaboração da solução.
%Esta seção é dividida em soluções de sistemas robóticos para os dois tipos
%de acessos, já que estes são o fator que mais restringe o desenvolvimento do
%sistema robótico por limitar suas dimensões, funcionalidades, e exigir a
%idealização conjunta de uma logística de acesso e movimentação do robô pelo aro
%câmara.
 
\subsection{Top hatch}
The top hatch is at the top of the arc chamber, it has only 35 cm diameter,
which is a challenge when thinking about using it as a robotic system access
point. Furthermore its proximity to the blades and the free space outside the
arc chamber create interesting possibilities for its use.
%Essa escotilha localizada no topo do aro câmara possui uma abertura de apenas
%35 cm de diâmetro, o que cria um desafio quando se pensa em utiliza-la como
%ponto de acesso para um robô. Por outro lado sua proximidade às pás e a área
% livre fora do aro câmara criam possibilidades interessantes para seu uso.

\textbf{Advantages}
\begin{itemize}
  \item robot fixation stability
  \item reference point, facilitating localization system, mapping, control and
  calibration
  \item built logistics: conveyor gantry to position the robot and the
  HVOF system
\end{itemize}

\textbf{Disadvantages}
\begin{itemize}
  \item difficulty in finding robots of such size (35 cm diameter)
  \item during the operation, the robot must be
  removed to rotate the blades
  \item not a general solution, specific to Jirau
\end{itemize}

The simplest solution to this access is to use an industrial robotic manipulator.
The choice of the commercial robot is primarily associated within reach. On the
other hand, only a small share of commercial robots have the dimension to fit
the top hatch. Thus, the study was focused on the KUKA Light Weight (LBR
iiwa 14 R820), robot whose base diagonal is less than 35 cm.

%A solução mais simples para este acessso é a utilização de um robô industrial
%comercial. A escolha do robô está primariamente associada ao seu alcance. Por
% outro lado, apenas uma pequena parcela dos robôs comerciais possuem a dimensão necessária para
%atravessar a escotilha. Sendo assim, o estudo foi direcionada para o uso do
%KUKA Light Weight (LBR iiwa 14 R820), robô cuja diagonal da base é inferior aos
%35 cm da escotilha.


The LBR R820 weighs 30 kg, has seven joints and has 14 kg payload, enough to
carry the coating equipment. However, further studies are needed to validate
the robot, as the speed and precision requirements when coating.

%O LBR R820 pesa 30 kg, possui 7 eixos e suporta carga de 14kg,
%suficiente por uma pequena margem para carregar o equipamento de
%revestimento. Entretanto, são necessários estudos aprofundados para valida-lo,
%como os requisitos de velocidade e precisão quando percorrendo a trajetória
%para a realização do revestimento.

To place the LBR R820 in a position where it is able to process all the
blade a hinged base model was proposed. The base consists of two links
interconnected by a revolute joint and is attached to the top hatch itself. To
cover the entire blade, the base must be able to assume different angles with
respect to the insertion axis, and the joint that connects the two segments of
the base and the tip of the base also need to move, which can be performed
manually or with prismatic actuators.

%Tendo como objetivo posicionar o LBR R820 em uma posição onde seja capaz de
%trabalhar toda a pá, um modelo de base articulada foi proposto. A base
%composta de dois elos interligados por uma junta de rotação é fixada na
%própria escotilha. Para que seja possível cobrir toda a pá,
%a base deve ser capaz de assumir diversas angulações com respeito ao
%eixo de inserção, e a junta que conecta os dois segmentos da base também
% precisa ter sua posição alterada, o que pode ser realizado manualmente ou com atuador.

Inserting the system, composed of the base and manipulator, in the top
hatch access requires special care, as the system total length is greater
than the distance from the top of the arc chamber to the turbine nose. Thus,
the arm and the base will need to be rotated during the insertion process,
which will result in misalignment of the central mass (with respect to the axis
perpendicular to the hatch) and will require a guide to compensate the torque
generated by such misalignment.

%Introduzir o robô, composto pelo conjunto base-LBR R820, é uma tarefa cuidadosa
%pois a extensão total será maior que a distância do topo do aro câmara ao
%cone da turbina. Ou seja, o braço e a base precisarão ser rotacionados
%durante o processo de inserção, o que acarretará no desalinhamento do centro de
%massa (com relação ao eixo perpendicular à escotilha) e exigirá uma guia para
%resistir ao torque gerado por esse desalinhamento.


As the system is attached to the top hatch and the joint base is fixed,
the torques are expected to be less than 3000 Nm on jointe base, and 4000 Nm on
the base during coating operation.

%Com o robô fixado na escotilha e a junta da base travada, são esperados torques
%inferiores a 3000 Nm sobre a junta e 4000 Nm sobre a base, durante a operação
% de revestimento.

%Ao se avaliar as possibilidades de realizar o revestimento sem a necessidade de
%placas de sacrifício, percebe-se que poderá ser realizado se a junta da
%base estiver paralela à pá (que para efeito de cálculo foi assumida com uma
%angulação de $45^o$ com relação ao eixo da turbina). Para esse caso a junta
% precisa realizar uma rotação à velcidade mínima de 34 rad/min. Essa velocidade possibilita a
%pistola de revestimento percorrer verticalmente a pá sem pausa. Percorrer
%horizontalmente a pá não será possível devido às restrições nas dimensções da
%base. O cálculo foi realizado por um estudo puramente geométrico, utilizando as
%informações reais de extensão da pá e área de trabalho do robô, assumindo um
%modelo simples da pá, sem curvatura.


\subsection{Bottom hatch}
The bottom hatch, localized at the bottom of the arc chamber, has 80 cm
diameter, enabling mid-sized robots entry, but it is 10 m far from the blade,
requiring system transportation and positioning. Using the bottom hatch as the
access for robot has the follwing advantages: large enough for medium sized
robots, free access, it is normally used by operators to turbine maintenance.
And the following disadvantages: not large enough for large robots, complex
logistics to the hatch (scaffolding and hoists), difficulty for robot moving
and positioning in the arc chamber due to the slippery and sloping floor.

%Solutions for the bottom hatch have the following advantages and
%disadvantages:
%Soluções que utilizem o acesso pela escotilha inferior, de dimensão maior,
%apresentam as seguintes vantagens e desvantagens:

%\textbf{Advantages}:
%\begin{itemize}
%  \item large enough for medium sized robots
%  \item free access
%  \item it is normally used by operators to turbine maintenance.
%\end{itemize}

%\textbf{Disadvantages}
%\begin{itemize}
%  \item not large enough for large robots
%  \item complex logistics to the hatch. Scaffolding and hoists are
%  needed
%  \item difficulty for robot moving and positioning in the arc chamber due to
%  the slippery and sloping floor.
%\end{itemize}

The bottom hatch, like all other accesses, is a logistical challenge and
a common challenge of the metallization process. The bottom hatch has 80 mm
diameter and is 4 m above ground. The system should be carried by a hoist
manually operated, installed inside the arc chamber on scaffolding. The floor
is slippery and has a cylindrical and inclined shape.

%O acesso pela escotilha inferior apresenta, como todos os outros
%acessos, um desafio logístico e o desafio comum do processo de metalização. O
%acesso à escotilha é realizado por uma abertura de 80 mm de diâmetro e 4 m
% acima do solo, logo os equipamentos são transportados por uma talha operada manualmente,
%instalada dentro do aro câmara, em andaimes. O solo é escorregadio e, devido à
%forma cilíndrica do aro câmara, curvilíneo e inclinado.

The system solutions were focused on medium size and lightweight manipulators,
due to robot shipping, handling and positioning of the robot. And modular base
solutions, when possible. Solutions were divided into subsections in accordance
with fixation strategies: mobile robots that move on rails fixed on the blade;
climbers; industrial manipulators that move on rails fixed on the floor.

%Dessa forma, as soluções foram focadas em robôs de médio porte, peso reduzido
%devido ao transporte e às necessidades de movimentação e posicionamento do robô
%(trajeto escotilha à pá), e modular, quando possível.

%As soluções foram divididas em subseções de acordo com a fixação:
%robôs móveis que se locomovem em trilhos, robôs escaladores e manipuladores
%industriais com base fixa. 
%obs.:
%Colocar o robô entre as pás para aplicar revestimento de duas pás com uma
% instalação exige que o robô seja desmontado toda vez que a turbina for girada.
%Isso é ruim, pois após primeira aplicação a área fica de risco. Esta solução
%exige 5 movimentos no robô e 6 movimentos na turbina.

%A solução em que o robô fica atrás ou à frente exige que o robô seja
%movimentado apenas 2 vezes e a turbina fará 8 movimentos (duas voltas).

%TODO revisar todos os projetos sabendo as respostas
\subsubsection{Design of a mobile robot and rail fixed on blade}\label{proj_rail}
The robotic manipulator and rail base fixed on blade satisfie all
requirements for the HVOF coating and the inspection process. The development of
a compact system for easy transportation and its installation on the arc
chamber are possible, since the manipulator dimensions are reduced due to the
extra mobility provided by the rail.

%A utlização de um manipulador robótico sobre trilhos satisfaz todos os
%requisitos para a realização de um processo de inspeção e metalização
%utilizando a técnica HVOF. O desenvolvimento de um sistema compacto para o
%transporte através do acesso pela escotilha inferior e sua instalação no aro
%câmara da turbina são possíveis, pois as dimensões do manipulador podem ser
%reduzidas por meio da mobilidade extra proporcionada pela introdução do trilho.

In the context of the proposed application, the solution consists in a system
similar to Roboturb, presented in section \ref{sec::rail}. Thus the rail should
be flexible to be able to follow the blade curvature, it should allow several placement options,
and, as the blade does not have a high magnetic permeability, the adhesion would
be by active suction cups with specific material to support large temperature
variations.

%No contexto da aplicação proposta, foram concebidas duas possibilidades para a
%fixação do sistema de trilhos. A primeira solução consiste em um sistema
%semelhante ao Roboturb, apresentado na seção \ref{sec::rail}. O sistema
% proposto se trata de um manipulador robótico com fixação diretamente na pá da
%turbina. O trilho deverá ser flexível para ser capaz de acompanhar a curvatura
%da pá e possibilitar diversas opções de posicionamento. Como o material da pá
%não possui alta permeabilidade magnética (Inox 420), a solução de fixação seria
%por ventosas ativas e com material específico para suportar as grandes
%variações de temperatura que a pá pode alcançar (temperatura ambiente a
%$100^oC$ durante a metalização).

%Uma abrangente pesquisa de robôs comerciais industriais de pequeno porte
% apontou que há manipuladores com carga entre 12 e 20 kg e velocidade necessários,
%sendo o LBR da Kuka o que possui melhor benefício peso/alcance, 30 Kg e 820 mm,
%respectivamente. %Para este manipulador, a metalização deverá ser
%realizada em, pelo menos, quatro etapas com quatro trilhos diferentes e
%customizados, e placas de sacrifício para evitar mau aplicação da metalização
%durante as trocas de sentido na movimentação do robô.

%A fixação de um trilho na pá apresenta diversas complexidades, como: a
%necessidade de manualmente instalar/desinstalar o sistema trilho/robô diversas
%vezes em cada pá; o projeto do trilho customizado e flexível; e ventosas ativas
%especiais que suportam variação de temperatura.




\textbf{Solution conclusion}

Fixing a rail on the blade has some complexities: rail and robot manual
installation/uninstallation for each blade side; design of customized flexible
rail; and design of special active suction cups that support temperature
variation. It is possible to use an industrial manipulator, such as the
Kuka lightweight (30 kg), making the design focused on signal processing,
mapping, localization, and control, and the rail construction. 

%A solução com trilho externo se mostrou vantajosa em comparação ao robô em
%trilho customizado acoplado à pá, devido à complexidade e intervenções
%manuais. Há a possibilidade de utilizar um manipulador industrial, tornando o
%foco do projeto em processamento de sinais, mapeamento, localização e controle,
%além da construção do trilho. Porém, a montagem da estrutura e a instalação de
%todo o sistema atrás da pá podem ser custosas, sendo esta ainda uma solução
%considerada complexa.
\subsubsection{Design of robotic climbers}\label{proj_climbers}
In this subsection, the robotic system solutions for HVOF coating are the fusion
of technologies documented in section~\ref{sota},
subsection~\ref{sota_climbers}. It will be an adaptation of \emph{The Climber},
ICM, given their ability to reconfiguration.

%Nesta subseção, consideram-se soluções para HVOF de pás de turbinas robôs
%escaladores com fusão das tecnologias documentadas na
%seção~\ref{sota}, subseção~\ref{sota_climbers}. Será abordada uma versão
%adaptada do robô \emph{The Climber}, ICM, dado sua possibilidade de
%reconfiguração.

\emph{The Climber}, ICM, is a commercial solution which meets many
of the HVOF specifications and enables improvement without compromising its
structure. The robot adhesion is by suction and locomotion by flexible mats.
The system has already been tested in hazardous environments, as wind turbines,
hydroelectric plants and others. We can divide the design into four systems:
mobility, adhesion, manipulator and autonomy.

%O robô \emph{The Climber}, ICM, é uma solução comercial que atende muitas das
%especificações HVOF e possibilita aperfeiçoamento sem comprometer sua
%estrutura. O robô possui sistema de adesão por sucção e locomoção através de
%esteiras flexíveis. O sistema já foi testado em ambientes de alta
%periculosidade, como turbinas eólicas, usinas hidrelétricas e outros. Podemos
%dividir o projeto em quatro sistemas: locomoção, adesão, manipulador e
% autonomia.

The system developed in \cite{kim2008development} has mats as locomotion type
and adhesion by suction. The system consists of pulleys, rubber belts, suction
cups, valves for each suction cup, DC motors for pulleys, control systems for
the valves and motors. \emph{The Climber} uses only one vacuum chamber instead
of suction cups, and the flexible mats allow smoothly and continuously motion.
The solution of a single camera seems more advantageous, as the robot can
move on curvatures up to 30 cm radius.

%O sistema desenvolvido em \cite{kim2008development} tem mecanismo de
%locomoção por esteiras e adesão por sucção. O sistema é composto por polias,
%correias de borracha, ventosas, válvulas para cada ventosa, motores DC para as
% polias, sistemas de controle para as válvulas e para os motores. \emph{The Climber}
%utiliza apenas uma câmara de vácuo, em vez de ventosas, e esteiras flexíveis
% que permitem maior suavidade e continuidade ao movimento. A solução por uma única
%câmara parece mais vantajosa, já que o robô consegue se locomover em curvaturas
%de até 30 cm de raio.

In the specific case of the HVOF process, a manipulator applies the coating 
while the robot travels along the blade. The robot locomotion on the blade rises
some design issues: the blade temperature during the procedure requires a
solution for suction active chamber with special material; mats and suction
chamber must work on accentuated curvatures.

%No caso específico da aplicação HVOF, o processo é realizado com
%manipulador enquanto o robô percorre a pá da turbina. A locomoção do
%robô sob a pá levanta algumas questões de projeto: a
%temperatura da turbina durante o procedimento exige uma solução por câmara
% ativa de material especial; e como se comporta o robô em curvaturas
%acentuadas. 


In adhesion by suction, an intelligent security mechanism should be implemented,
with accelerometers, gyros and other sensors to ensure the shutdown of the
electronics and the supply of the HVOF gases. The solution of a mobile robot
path planning increases safe operation and the optimal control of the adhesion
mechanism can limit the maximum suction force.

%Em sistemas de adesão por sucção, deve-se considerar um mecanismo
%inteligente de segurança, possivelmente utilizando acelerômetros e outros
%sensores, para garantir o desligamento do sistema eletrônico e o fornecimento
% de gáses. A solução de um robô móvel com planejamento de trajetória aumenta a
%segurança da operação e o controle ótimo do mecanismo de adesão pode limitar a
% força máxima de sucção.

The manipulator to be designed for HVOF application should have has the
following characteristics: to be lightweight, avoiding complex adhesion and
balance; fast and accurate as required by the HVOF application; modular because
the operation will be performed \textit{in situ}, in confined spaces;
to have small dimensions, improving mobility, but sufficient to operate
blade edges considering the 230 mm minimum distance between HVOF gun and
blade; and payload and vibration capacity.

%O manipulador a ser projetado para aplicação HVOF possui as seguintes
%características: é leve para não comprometer a adesão e equilíbrio do sistema
%móvel; rápido e preciso conforme requer a aplicação HVOF; modular, já
%que a operação será realizada in-situ, em espaço confinado;
%não possui grandes dimensões, pois o robô é móvel e pode
%percorrer a pá, porém deve ser suficiente para operar em pontos de
%difícil acesso à base e considerar a distância mínima (230 mm) entre pistola
%HVOF e pá; e é capaz de sustentar a carga e vibrações geradas pela
%pistola HVOF. 

The path planning solution should consider both the mobile base and the
manipulator. The literature is fairly consolidated on robotic manipulators, and
many of these problems are already settled and available, as developed in
\cite{manzdevelopment}. The smaller industrial manipulators with required
payload have 30 to 50 kg. Therefore, manipulator, HVOF gun and cables could have
50 to 80 kg.

%A solução de robôs escaladores exige planejamento de trajetórias tanto da base
%móvel, quanto ao controle de manipuladores. A literatura sobre
%manipuladores é bastante consolidada, sendo muitos dos problemas citados já
%resolvidos e disponíveis no mercado, como o desenvolvido em
%\cite{manzdevelopment}. Os menores manipuladores industriais que sustentam a
%carga do sistema de metalização possuem em torno de 30 a 50 kg. Portanto, o
%conjunto manipulador, pistola e cabos pode possuir de 50 a 80 kg de massa.

%A tecnologia que verifica a necessidade de
%revestimento, com sensores laser e ultrassom, e poderá indicar o \textbf{mapa
%ou apenas realizar um teste de sucesso/falha} \citep{escaler2006detection}.

The autonomy of a mobile robot is the robot intelligence. It covers the mission
control, i.e., the planning and execution of tasks. The locomotion is performed
by motors control together with the active suction control, the path planning,
the obstacle avoidance and environmental mapping, through a set of sensors such
as accelerometers and laser. The manipulator control can be cinematic by
visual servoing or structured environment. And a vehicle support system will
be responsible for security, smooth operation and power management.

%O sistema autônomo de um robô móvel é a inteligência do robô. Ele abrange o
%controle de missão, ou seja, o planejamento e execução das tarefas em modo
%autônomo. A locomoção será realizada pelo controle dos motores em conjunto com
% o controle do sistema ativo de adesão por sucção, o planejamento de trajetória, desvio de
%obstáculos e mapeamento do ambiente, através de um conjunto de sensores, como
%laser e acelerômetros. O controle do manipulador poderá ser cinemático por
%servovisão ou pela estruturação do ambiente. E um sistema de suporte do veículo
%ficará responsável pela segurança, bom funcionamento e gerenciamento de
% potência do robô.


The climber as described above does not have automatic switching between blades.
A climber with arms to switch between blades is a costly solution in terms
of control and mechanical structure. Another solution would be a robot with
locomotion by sliding segments, as RRX3, and adhesion by suction, but the
flexibility required for locomotion between blades and the distance between
turbines complicates the design. Thus, the exchange between blades should be
manual.

%As características descritas acima como solução de um robô escalador impede a
%troca automática entre pás. Um robô escalador com tecnologia de avanço
% pendurado por braços é uma solução muito custosa em termos de controle e estrutura
%mecânica. Outra solução seria um robô com locomoção por segmentos deslizantes,
%como o RRX3, e adesão por sucção, porém a flexibilidade exigida para a
% locomoção entre pás e a distância entre turbinas complexifica o projeto. Dessa forma, a
%troca entre pás deverá ser manual.

%\textbf{Versão adaptada Roboturb}

%O Roboturb, como já descrito na subseção~\ref{sec::rail}, é um
%manipulador que se locomove em um trilho, este acoplado à pá da turbina
%por ventosas (sucção). A solução não permite a extensão
%do manipulador, já que o peso desequilibra a estrutura e não há torque para
%compensar a força exercida no efetuador durante a operação HVOF. A segunda
%solução de robôs escaladores é adicionar um trilho perpendicular e transformar
%o Roboturb em um robô móvel, com locomoção através de dois trilhos, idéia
%semelhante ao \emph{Climbing robot for Grit Blasting}, que utiliza duas
%plataformas deslizantes com ventosas.

%Os trilhos são compostos por esteiras flexíveis nas extremidades para a
%locomoção, como \emph{The Climber}, e as ventosas são ativas e distribuídas por
%todo o trilho. O manipulador só necessitaria mover em um dos trilhos para
%percorrer toda a pá, já que os trilhos também se movimentam. 

%A solução de trilhos móveis com manipulador é dependente à curvatura da pá da
%turbina e o aumento da flexibilidade do trilho para se locomover sob a pá pode
%impedir a movimentação do manipulador. Dessa forma, é considerada uma solução
%muito específica e restrita à aplicação.

\textbf{Solution conclusion}
Although tempting because of the autonomy, the complexity of blade structure,
the environment, the required speed and payload are
major challenges to the design. It is estimated a load of 50 kg, which greatly increases the
dimensions of the mobile base and hence diminishes their workspace,
slowing the process.

%Apesar de tentador devido à autonomia, a complexidade da estrutura da pá, o
%ambiente, a velocidade requerida e a carga do sistema de metalização são
%grandes desafios ao projeto. São estimados 50 kg de
%carga para o conjunto manipulador, cabos e pistola, o que aumenta muito as
%dimensões da base móvel e, consequentemente, diminui a sua área de atuação,
%tornando o processo mais demorado.
\subsubsection{Design of industrial manipulator with rail on
floor}\label{proj_manip}

There are several industrial manipulators which fulfill the HVOF requirements.
The companies Fanuc, Motoman, ABB and KUKA produce manipulators with compatible
dimensions to the bottom hatch; speed, accuracy, and workspace that meet the
requirements for HVOF process; and manipulators which can coat all over one side
of the blade, in a fixed base position. However, those manipulators are too big
to operate behind the blade, on the distributor side, due to collision and
joint constraints/robot manipulability. Besides, manipulators with such workspaces are
heavyweight, thus the robot placement and locomotion inside the arc chamber
would be complex. Therefore, the logic solution would be to select a mid-sized
manipulator and made a rail type base to place and move the robot. 

%Há diversos manipuladores robóticos industriais com as especificações
%necessárias para a realização da tarefa de metalização por HVOF. As empresas
%Fanuc, Motoman, ABB e KUKA fabricam manipuladores com dimensões compatíveis com
% o acesso pela escotilha inferior e velocidade, precisão, e espaço de trabalho que
%cumprem os requisitos para a execução do processo em todo um lado da pá, em uma
%base fixa. Porém, há incompatibilidade atrás da pá e a necessidade de escolher
% a posição correta do manipulador em relação à pá, a fim de maximizar a sua área de trabalho, no ambiente da
%turbina, o que pode restringir os seus movimentos.
%Como as pás podem ser giradas até um ângulo de $14.5^o$, são discutidas as
% ideias de posicionamento do manipulador entre as pás, a fim de executar a operação em ambos os lados da
%pá (um lado de cada pá), e o posicionamento fixo à frente e depois atrás à pá.

The base consists of a transport rail and a positioning rail, both are fixed
in the arc chamber by magnetic coupling and/or welding. The first rail starts at
the bottom hatch entry and goes to the blade, allowing the locomotion of the
robot in the arc chamber. The latter is coupled to the main track and positions
the robot close to the blade, enabling displacement along it. Thus, the
base may be summarized in three joints: prismatic, revolution and prismatic (PRP). As the robot
can not not reach the entire blade, there is still the need of different
vertical positions, which should be manually selected. The blade can coat in
linear or circular motion, and, in both cases, the manipulator will be
responsible for the speed, position and gun orientation. The gun direction
exchange should occur outside the blade, or sacrifice plates should be used, or
valves to redirect the coating particles should be used.

%A base por um trilho de transporte e um de posicionamento. Ambos são fixados no
%aro câmara por acoplamentos magnéticos e/ou solda. O primeiro começa na
%entrada da pá e vai até a pá, permitindo o deslocamento do robô pelo aro
% câmara.
%O último é acoplado ao trilho principal e posiciona o robô próximo a pá, além
% de possibilitar o deslocamento ao longo desta. Dessa forma, a base pode ser
%resumida em três juntas: prismática, revolução e prismática (PRP). Como o robô
%não possui alcance de toda a pá, há, ainda, a necessidade de posições verticais
%diferentes escolhidas manualmente. A pá pode ser processada em movimentos
%circulares ou lineares e, em ambos os casos, o manipulador ficará responsável
% pela velocidade, posição e orientação do processo. A troca de sentido de movimento deverá ocorrer fora da
%pá, ou devem ser utilizadas placas de sacrifício, ou válvulas para
%redirecionamento das partículas de revestimento.

%Esse tipo de abordagem simplifica a movimentação do robô no
%trilho, uma vez que o trilho seria totalmente reto, e possibilitaria a
%metalização de um dos lados das quatro pás com uma única instalação de base.
%Porém, mesmo nesta solução, a altura do trilho deverá ser ajustada três vezes
% para cada lado de pá.

It is also required in this design the implementation of a
localization/calibration system, as the robot must know its location in
relation to the blade for autonomous operation. The localization requires
external sensors, as cameras, or installed on manipulator tip, or base.

%Em ambos os sistemas propostos, é necessária a implementação de um sistema de
%localização do robô em relação à pá, tornando possível a geração de um
%planejamento de trajetórias para o processo de metalização. O sistema de
%localização pode ser concebido por sensores externos
%ao robô (câmeras e outros), ou instalados no próprio manipulador/base.


%The alternative to avoid contact with the blade consists of a single rail
%rectilinear fixed by magnetic bases or welding of the rim in the soil chamber.
% Like the robot has not reach all the shovel, there is still a need positions
%different vertical. The blade can be processed in circular movements or
%linear and, in both cases, the handler will be responsible for the speed,
%position and orientation of the process. The exchange direction of movement
% should occur outside the shovel or must sacrifice cards used.

\textbf{Transport rail}

The figure~\ref{fig::andaime} shows the space between the blades inside the
arc chamber. A mid-sized robot manipulator should be hoisted through the bottom
hatch, placed on a rail type base with magnetic holders and then should be moved
to that space, as in figura~\ref{fig::rail1}. At that position, the robot should
switch to the other rail. This position (robot between blades) is advantageous because it
allows the manipulator to operate two blades (front and back of one of the
other) without rail disassembling or major positioning changes.

\begin{figure}[h!]	
	\includegraphics[width=\columnwidth]{figs/manipuladores/rail1.PNG}
	\caption{Transport rail}
	\label{fig::rail1}
\end{figure}

%A figura~\ref{fig::andaime} mostra o espaço entre as pás da turbina, dentro do
%aro câmara. Um robô manipulador de médio porte pode ser fixado em uma base
%magnética, na posição que se encontra a escada da figura~\ref{fig::andaime}.
%Essa posição é vantajosa por possibilitar a execução da tarefa em duas pás
%(frente de uma e verso da outra), sem desmontar ou fazer grandes alterações no
%posicionamento da base do robô, diminuindo as intervenções e tempo de tarefa.

A purely geometric study shows that the workspace of a robotic manipulator
for processing both blades, assuming a fixed base between the blades, should be
around 5 m. The industrial manipulator IRB5500, for example, developed by ABB
for painting, is a large sized manipulator and has 3 meters range, 600 kg,
which makes it very difficult or even impossible to \textit{in situ} logistics.
As no industrial robot was found in requirements, a positioning rail should be
made.

%O estudo puramente geométrico demonstra que o alcance do manipulador robótico
%para o processamento de ambos os lados das pás, considerando uma base fixa
% entre as pás, deverá ser em torno de 5 metros. O manipulador industrial IRB5500,
%desenvolvido pela ABB para pintura, possui 3 metros de alcance, porém 180 kg, o
% que já dificulta ou até impossibilita a logística de movimentação e posicionamento in-situ. Não foi encontrado um robô
%industrial com o alcance necessário e que tivesse as dimensões máximas da
%escotilha inferior. 

%A solução conceitual de posicionar um manipulador industrial entre as pás deve
%avaliar, portanto, todas as configurações necessárias da base (orientações e
%posições) para garantir que todo o espaço de trabalho do manipulador mais base
%cubra os lados de ambas as pás. O número de configurações e o projeto
%mecânico da base são necessários para a viabilização da solução,
%uma vez que será possível avaliar as intervenções e complexidades. Bases
%autônomas diminuem o número de intervenções e aumentam a precisção do sistema,
%porém aumentam a complexidade, o custo devido ao número de sensores e
% atuadores, e o peso do sistema, prejudicando a logística.

\textbf{Positioning rail}
Fixedly positioning a robotic manipulator with magnetic holders at the front or
at the back of the blade for HVOF process is a natural solution, since it is
similar to the Rijeza and other companies procedure. However, a purely geometric
study, using the real dimensions of the blade, shows that the manipulator must
have more than 2.0 m range and 1.0 m height. Workspace analysis, kinematics and
dynamics studies, and collision detection should be conducted to confirm the
geometric study. A mid-sized robot can not reach the entire blade, thus
its base should be able to move horizontally and vertically
(figura~\ref{fig::rail2}).

\begin{figure}[h!]	
	\includegraphics[width=\columnwidth]{figs/manipuladores/rail2.PNG}
	\caption{Positioning rail}
	\label{fig::rail2}
\end{figure}
%Posicionar de maneira fixa um manipulador com base magnética à frente e atrás
% da pá para a metalização é uma solução natural, já que é semelhante à utilizada pela
%empresa Rijeza atualmente. Um estudo puramente geométrico, utilizando as
%dimensões da pá, mostra que o manipulador deve possuir alcance de 1.7 m e ser
%posicionado a uma altura de 1.1 m em relação ao solo. Estudos de espaço de
%trabalho, manipulabilidade e colisões devem ser realizados para confirmar o
%estudo geométrico.

In this solution, a robot placed on a rail base on the floor can not coat
blades on top, thus the turbine must be manually rotated and a new calibration process should be done
for each step.

%O posicionamento do sistema à frente ou atrás da pá exige
%intervenções para rotação da turbina e para o deslocamento do sistema. Em
% relação a um sistema com base autônoma entre as pás, o processo parece mais custoso em
%intervenções manuais e mais demorado, porém bem mais simples em termos de
%robótica.

\textbf{Solution conclusion}
In terms of robotics, industrial manipulators on a rail base is the simplest
solution among all solutions to access through the bottom hatch. There is no
mechanical design for the manipulator, since it will be acquired in one of the
aforementioned manufacturers, but the mechanics will be responsible for the base
design and logistics. In addition, the following steps should be done:
manipulator control, data processing, calibration, path planning and UI.

%A utilização de manipuladores industriais é a mais simples, em termos de
%sistemas robóticos, dentre todas as soluções para o acesso pela escotilha
%inferior.

%Não há projeto mecânico do manipulador, já que este será adquirido em um dos
% fabricantes citados. As dificuldades mecânicas do projeto serão em relação à logística de posicionamento e movimentação do robô dentro do aro câmara, e no desenvolvimento de uma base,
%que pode ser autônoma. Além disso, o projeto fica responsável pelo controle do
%manipulador, processamento de dados que envolvem o HVOF, planejamento de
%trajetórias e UI.

%The main challenge is to build a rigid base and the locomotion of
%Equipment for the ring chamber. This conceptual design will be one of fronts to
%the feasibility study.

%Os desafios consistem na construção de uma base rígida e a locomoção dos
%equipamentos pelo aro câmara. Este projeto conceitual será uma das frentes para
%o estudo de viabilidade.
 
%\input{proj_riwea}
%\input{proj_acesso_grande}
\subsection{Conceptual solution}
As a conclusion of the proposals, the concept solution is to use a
industrial manipulator on a base. The characteristic of the manipulator, and the base
varies with the hatch (top or bottom). If top hatch, the solution is a small
sized industrial manipulator and custom base electronically operated; in
the case of the bottom hatch, the solution is a mid-sized industrial manipulator
with rail base and magnetic holders.

%Como conclusão das propostas, a solução conceito é a utilização de um
%manipulador industrial sobre uma base. A característica do manipulador e da
% base varia de acordo com o ponto de acesso: no caso
%da escotilha superior, a solução é um manipulador industrial de pequeno porte e
%base customizada operada eletronicamente; no caso da escotilha inferior,
%manipulador industrial de porte médio e base magnética; no último caso de
% acesso pela jusante, será escolhido um manipulador industrial de grande porte com base
%fixa magnética. 
%\section{Estudo de bases para manipualdores industriais}

\input{estudo_solid} 
\input{estudo_base}
\section{Conclusion and future work}\label{sec:conclusions}
The maintenance of hydroelectric turbines is essential for hydropower plants
fully operation, as it substantially increases the power plant potential. The
maintenance of the hydraulic profile of turbine blades is a major concern for
turbine efficiency, thus regular inspections, repairs and coating application
for cavitation and abrasion protection should be done.

The current hard coating operation is costly, as it requires turbine
disassembling, thus this document aims: to analyze the constraints of the
\textit{in situ} thermal spray coating process; to characterize the environment
where the process is taking place; to make a detailed state of the art study
of similar problems; and to design conceptual solutions.

%Este documento teve como objetivo: fazer uma análise das restrições do processo
%de revestimento por aspersão térmica; caracterizar o ambiente de trabalho onde
% o processo será realizado; fazer um estudo detalhado do estado da arte que
%visaram solucionar um problema semelhante ou possuíam tecnologias que
%poderiam ser utilizadas como solução; apresentar soluções conceituais; e
%fazer um estudo de viabilidade técnica para as soluções. 

The feasibility study for an \textit{in situ} coating application is promising
and some possible solutions were investigated for each turbine access. All
solutions run into some logistical and technical challenges, which will be
detailed at the development of EMMA project. For future work, the
mechanics, calibration, control, and user interface systems should be fully
detailed.

%O estudo de viabilidade de uma solução para revestimento \textit{in situ} se
%mostrou promissor e foram apontadas algumas possíveis soluções considerando
% cada acesso ao aro câmara da turbina. Todas as soluções esbarram em alguns desafios
%logísticos e técnicos que serão abordados detalhadamente até o fim do projeto
%EMMA. Os projetos de bases mecânicas para as diversas soluções serão abordados,
%assim como suas instalações, manuseio e posicionamento. Além disso, toda a
% parte de localização, calibração e mapeamento realizado pelo robô, seu controle e
%interface de usuário ainda serão desenvolvidos.

  
\bibliography{main} 
\appendix
\end{document}
